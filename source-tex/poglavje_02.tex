\documentclass[kapital_02.tex]{subfiles}

\begin{document}

Krožni\KPEstran tok produktivnega kapitala ima splošni obrazec: \( \KPEP \KPEpike \KPEB' \KPEcrta \KPED' \KPEcrta \KPEB \KPEpike \KPEP \). Pomeni periodično obnavljano funkcijo produktivnega kapitala, torej reprodukcijo, ali njegov produkcijski proces kot proces reprodukcije glede na večanje vrednosti; ne le produkcijo, ampak periodično produkcijo presežne vrednosti; funkcijo industrijskega kapitala v njegovi produktivni obliki, ne kot enkratno funkcijo, ampak kot periodično ponavljajočo se, tako da določa ponovni začetek izhodišče s\^amo. Del \( \KPEB' \) lahko (v nekaterih primerih, v nekaterih panogah, v katere nalagajo industrijski kapital) takoj vstopi znovič v isti produkcijski proces, iz katerega je izšel kot blago; s tem doseže, da njegove vrednosti ni treba pretvoriti v dejanski denar ali denarni znak, oziroma samostojen izraz pridobi le kot računski denar. Ta del vrednosti ne gre v cirkulacijo. V produkcijski proces prihajajo tako vrednosti, ki ne pridejo v proces cirkulacije. Isto velja za tisti del \( \KPEB' \), ki ga kapitalist porabi kot del presežnega produkta in natura. Vendar to za kapitalistično produkcijo ni pomembno; v poštev pride kvečjemu v poljedelstvu.

Pri tej obliki je takoj videti dve stvari.

Prvič. Medtem ko produkcijski proces, funkcija \( \KPEP \), v prvi obliki \( \KPED\KPEpike\KPED' \) prekinja cirkulacijo denarnega kapitala in nastopa kot posredovalec med obema njenima fazama \( \KPED\KPEcrta\KPEB \) in \( \KPEB'\KPEcrta\KPED' \), je celotni proces cirkulacije industrijskega kapitala, celotno njegovo gibanje v fazi cirkulacije, tu samo prekinitev in zato samo posredovanje med produktivnim kapitalom, ki\KPEstran prične krožni tok kot prvi ekstrem in ki ga kot zadnji v isti obliki, torej v obliki njegovega ponovnega pričetka, sklene. Prava cirkulacija je le posredovanje periodično obnavljane in z obnavljanjem brez prestanka potekajoče reprodukcije.

Drugič. Celotna cirkulacija nastopa v obliki, ki je nasprotna tisti, katero ima v krožnem toku denarnega kapitala. Tam je bila: \( \KPED\KPEcrta\KPEB\KPEcrta\KPED \) (\( \KPED\KPEcrta\KPEB \). \( \KPEB\KPEcrta\KPED \)), če pustimo ob strani velikost vrednosti; tu je, če spet pustimo ob strani velikost vrednosti, \( \KPEB\KPEcrta\KPED\KPEcrta\KPEB \) (\( \KPEB\KPEcrta\KPED \). \( \KPED\KPEcrta\KPEB \)), torej oblika enostavne cirkulacije blaga.

\section{Enostavna reprodukcija}

Poglejmo torej najprej proces \( \KPEB'\KPEcrta\KPED'\KPEcrta\KPEB \), ki poteka med ekstremoma \( \KPEP\KPEpike\KPEP \) v sferi cirkulacije.

Izhodiščna točka te cirkulacije je blagovni kapital: \( \KPEB' = \KPEB + \KPEb = \KPEP + \KPEb \). Funkcijo blagovnega kapitala \( \KPEB'\KPEcrta\KPED' \) (realizacijo v njem vsebovane kapitalske vrednosti \( \KPEP \), ki obstaja sedaj kot sestavni del blaga \( \KPEB \), kakor tudi v njem vsebovane presežne vrednosti, ki obstaja kot sestavni del iste blagovne množine v vrednosti \( \KPEb \)) smo opazovali v prvi obliki krožnega toka. Vendar je tvorila tam drugo fazo prekinjene cirkulacije in sklepno fazo celotnega krožnega toka. Tu tvori drugo fazo krožnega toka, toda prvo fazo cirkulacije. Prvi krožni tok konča z \( \KPED' \); ker pa lahko začne \( \KPED' \) prav tako kakor začetni \( \KPED \) na novo drugi krožni tok kot denarni kapital, najprej ni bilo treba raziskovati, ali nadaljujeta v \( \KPED' \) vsebovana \( \KPED \) in \( \KPEd \) (presežna vrednost) svojo pot skupaj ali pa ubirata različni poti. To bi bilo nujno samo, če bi nadalje zasledovali prvi krožni tok, ko se je obnovil. Toda to vprašanje se mora rešiti v krožnem toku produktivnega kapitala, ker je od tega odvisna že določitev njegovega prvega krožnega toka in ker je v njem \( \KPEB'\KPEcrta\KPED' \) prva faza cirkulacije, ki se mora dopolniti z \( \KPED\KPEcrta\KPEB \). Od te odločitve je odvisno, ali predstavlja obrazec enostavno reprodukcijo ali reprodukcijo v razširjenem obsegu. Po tem, ali se odloči za eno ali drugo, se spremeni značaj krožnega toka.

Vzemimo\KPEstran torej najprej enostavno reprodukcijo produktivnega kapitala, pri čemer predpostavljajmo tako kakor v prvem poglavju nespremenjene okoliščine in nakup in prodajo blaga po njegovi vrednosti. Vsa presežna vrednost gre po tej predpostavki v osebno porabo kapitalista. Kakor hitro se spremeni blagovni kapital \( \KPEB' \) v denar, kroži tisti del denarne vsote, ki predstavlja kapitalsko vrednost, naprej v krožnem toku industrijskega kapitala; drugi del, ki je vnovčena presežna vrednost, gre v splošno blagovno cirkulacijo, je denarna cirkulacija, ki se začne pri kapitalistu, poteka pa zunaj cirkulacije njegovega kapitala.

V našem primeru smo imeli blagovni kapital \( \KPEB' \), sestoječ iz \( 10000 funtov \) preje v vrednosti 500~f.~št.; od tega so vrednost produktivnega kapitala in nadaljujejo kot denarna oblika 8440 funtov preje cirkulacijo kapitala, ki jo je začel \( \KPEB' \); presežna vrednost 78~f.~št., denarna oblika 1560 funtov preje, presežnega dela blagovnega produkta, pa izstopa iz te cirkulacije in opisuje ločeno pot v okviru splošne cirkulacije blaga.

\[
    \KPEB'
    \left\lbrace
    \begin{array}{c}
        \KPEB \\
        + \\
        \KPEb \\
    \end{array}
    \right\rbrace
    \begin{array}{c c}
        \KPEcrta & \KPEcrta \\
        \KPEcrta & \KPED' \\
        \KPEcrta & \KPEcrta \\
    \end{array}
    \left\lbrace
    \begin{array}{c}
        \KPED \\
        + \\
        \KPEd \\
    \end{array}
    \right\rbrace
    \begin{array}{c l}
        \KPEcrta & \KPEBrazcepDsPs \\
         & \\
        \KPEcrta & \KPEb \\
    \end{array}
\]
% \begin{tikzpicture}[]

%     \matrix[matrix of math nodes]{
%         & \node (\KPEB) {\KPEB}; & & & \node (\KPED) {\KPED}; & \KPEcrta & \KPEB & \KPErazcep^{\KPEDs}_{\KPEPs} \\
%         \KPEB' & + & \KPEcrta & \KPED' & + & & \\
%         & \node (\KPEb) {\KPEb}; & & & \node (\KPEd) {\KPEd}; & \KPEcrta & \KPEb \\
%     };

%     \draw [] (\KPEB.east) -- (\KPED.west);
%     \draw [] (\KPEb.east) -- (\KPEd.west);
%     \draw [decorate,decoration={brace, mirror}] (\KPEB.west) -- (\KPEb.west);
%     \draw [decorate,decoration={brace}] (\KPEB.east) -- (\KPEb.east);
%     \draw [decorate,decoration={brace, mirror}] (\KPED.west) -- (\KPEd.west);
%     \draw [decorate,decoration={brace}] (\KPED.east) -- (\KPEd.east);
% \end{tikzpicture}

\( \KPEd \KPEcrta \KPEb \) je vrsta nakupov z denarjem, ki ga troši kapitalist bodisi za blago v pravem pomenu besede, bodisi za storitve v prid svoji cenjeni osebi ali družini. Ti nakupi so razdrobljeni, izvršeni ob različnih časih. Denar obstaja torej začasno v obliki denarne zaloge ali zaklada, ki je določen za tekočo konsumpcijo, ker ima denar, katerega cirkulacija je prekinjena, obliko zaklada. Njegova funkcija cirkulacijskega sredstva, ki vključuje tudi njegovo prehodno obliko zaklada, ne gre v cirkulacijo kapitala v njegovi denarni obliki \( \KPED \). Denarja ne zalaga, ampak ga troši.

Predpostavili smo, da prehaja celotni založeni kapital vedno ves iz ene svoje faze v drugo. Tako predpostavljamo tudi tu, da vsebuje blagovni produkt \( \KPEP \) celotno vrednost produktivnega kapitala \( \KPEP \) = 422~f.~št. plus v produkcijskem procesu ustvarjeno presežno vrednost = 78~f.~št. V našem primeru, v\KPEstran katerem imamo opravka s poljubno deljivim blagovnim produktom, obstaja presežna vrednost v obliki 1560 funtov preje; prav tako kakor obstaja, preračunana na 1 funt preje, v obliki 2,496 unč preje. Če pa bi bil blagovni produkt na primer stroj, vreden 500~f.~št. in enake vrednostne sestave, bi bil sicer del vrednosti tega stroja 78~f.~št. presežna vrednost, toda teh 78~f.~št. bi obstajalo samo v celem stroju; stroja ni mogoče razdeliti na vrednost kapitala in presežno vrednost, ne da bi ga razbili na kose in tako uničili z njegovo uporabno vrednostjo tudi njegovo vrednost. Oba sestavna dela vrednosti je torej mogoče izraziti v sestavnih delih blagovnega produkta le idealno, ne kot samostojna elementa blaga \( \KPEB' \), kakor je vsak funt preje ločljiv, samostojen blagovni element 10.000 funtov preje. V prvem primeru je treba celotno blago, blagovni kapital, stroj, v celoti prodati, preden lahko začne \( \KPEd \) svojo posebno cirkulacijo. Če pa proda kapitalist 8440 funtov preje, bi pomenila prodaja nadaljnjih 1560 funtov preje popolnoma ločeno cirkulacijo presežne vrednosti v obliki \( \KPEb \) (1560 funtov preje) \( \KPEcrta \) \( \KPEd \) (78~f.~št.) \( \KPEcrta \) \( \KPEb \) (potrošni predmet). Vrednostne elemente vsakega posameznega deleža produkta 10.000 funtov preje pa prav tako lahko nakažemo v delih produkta kakor v celotnem produktu. Tako kot lahko razdelimo le-tega, tj. 10.000 funtov preje, v vrednost konstantnega kapitala (\( \KPEc \)), 7440 funtov preje v vrednosti 372~f.~št., v vrednost variabilnega kapitala (\( \KPEv \)), 1000 funtov preje z vrednostjo 50~f.~št., in v presežno vrednost (\( \KPEpv \)), 1560 funtov preje z vrednostjo 78~f.~št., se lahko razdeli vsak funt preje v \( \KPEc \) = 11,904 unč z vrednostjo 8,928 penija, \( \KPEv \) = 1,600 unč preje v vrednosti 1,200 penija, \( \KPEpv \) = 2,496 unč preje z vrednostjo 1,872 penija. Kapitalist bi lahko tudi pri postopni prodaji 10.000 funtov postopoma potrošil v zaporednih deležih vsebovane sestavine presežne vrednosti in s tem prav tako postopoma realiziral tudi vsoto \( \KPEc + \KPEv \). Konec koncev pa tudi ta operacija zahteva, da proda vseh 10.000 funtov, da nadomesti s prodajo 8440 funtov torej tudi vrednost \( \KPEc \) in \( \KPEv \) (I. knjiga, 7. poglavje, 2).

Najsi je to tako ali drugače, tako v \( \KPEB' \) vsebovana kapitalska vrednost kakor tudi presežna vrednost pridobita z \( \KPEB' \KPEcrta \KPED' \)\KPEstran obstoj, v katerem se lahko razločita, obstoj različnih denarnih vsot; v obeh primerih sta tako \( \KPED \) kakor tudi \( \KPEd \) resnično spremenjeni obliki vrednosti, ki ima prvotno v \( \KPEB' \) lastni izraz le kot cena blaga, torej le idealni izraz.

\( \KPEb \KPEcrta \KPEd \KPEcrta \KPEb \) je enostavna cirkulacija blaga, katere prvo fazo \( \KPEb \KPEcrta \KPEd \) vsebuje cirkulacija blagovnega kapitala \( \KPEB' \KPEcrta \KPED' \), torej krožni tok kapitala, katere dopolnilna faza \( \KPEd \KPEcrta \KPEb \) pa sodi izven tega krožnega toka kot od njega ločeno dejanje splošne cirkulacije blaga. Cirkulacija \( \KPEB \) in \( \KPEb \), kapitala in presežne vrednosti, se po spremembi \( \KPEB' \) v \( \KPED' \) razcepi. Iz tega izhaja:

1.\ Ker se z \( \KPEB' \KPEcrta \KPED' = \KPEB' \KPEcrta ( \KPED + \KPEd ) \) realizira denarni kapital, se gibanje kapitalske vrednosti in presežne vrednosti, ki je v \( \KPEB' \KPEcrta \KPED' \) še skupno in vsebovano v isti blagovni množini, lahko odcepi drugo od drugega, ker imata zdaj obe kot denarni vsoti samostojni obliki.

2.\ Če se kapitalska vrednost in presežna vrednost razcepita, ker troši kapitalist \( \KPEd \) kot dohodek, medtem ko nadaljuje \( \KPED \) kot funkcionalna oblika kapitalske vrednosti svojo pot, kakor jo določa krožni tok -- je mogoče prikazati prvi akt \( \KPEB' \KPEcrta \KPED' \) v zvezi z naslednjima aktoma \( \KPEB \KPEcrta \KPED \) in \( \KPEb \KPEcrta \KPEd \) kot dve različni cirkulaciji: \( \KPEB \KPEcrta \KPED \KPEcrta \KPEB \) in \( \KPEb \KPEcrta \KPEd \KPEcrta \KPEb \); po svoji splošni obliki pripadata obe zaporedji navadni blagovni cirkulaciji.

Sicer pa se v praksi sestavni deli vrednosti pri nedeljivih blagovnih telesih, ki jih ni mogoče razcepiti, izolirajo miselno vsak zase. V londonskem stavbarstvu na primer, ki dela večinoma na kredit, dobiva stavbenik glede na različne faze, v katerih je gradnja hiše, predujme. Nobena od teh faz ni hiša, ampak le dejansko obstoječ sestavni del nastajajoče hiše; kljub svoji realnosti torej le idealni del celotne hiše, pa vendar zadosti realen, da lahko služi za zavarovanje dodatnega predujma. (Glej o tem v 12. poglavju.)

3.\ Če se gibanje kapitalske vrednosti in presežne vrednosti, ki je v \( \KPEB' \) in \( \KPED' \) še združeno, razdvoji samo deloma (tako, da se del presežne vrednosti ne potroši kot dohodek) ali pa sploh ne, tedaj se kapitalska vrednost spremeni še med svojim krožnim tokom, še preden se le-ta konča. V našem primeru\KPEstran je bila vrednost produktivnega kapitala enaka 422~f.~št. Če nadaljuje kroženje \( \KPED \KPEcrta \KPEB \), na primer kot 480~f.~št. ali 500~f.~št., preteče zadnje stadije krožnega toka kot za 58~f.~št. ali 78~f.~št. večja vrednost, kakor je bila spočetka. To je lahko hkrati združeno s spremembo njegove vrednostne sestave.

\( \KPEB' \KPEcrta \KPED' \), drugi stadij cirkulacije in sklepni stadij krožnega toka I (\( \KPED \KPEpike \KPED' \)), je v našem krožnem toku drugi stadij le-tega in prvi stadij blagovne cirkulacije. Kolikor gre torej za cirkulacijo, se mora dopolniti z \( \KPED' \KPEcrta \KPEB' \). Toda \( \KPEB' \KPEcrta \KPED' \) nima že za seboj le procesa večanja vrednosti (tu funkcijo \( \KPEP \), prvi stadij), ampak je že vnovčen tudi njegov rezultat, blagovni produkt \( \KPEB' \). Proces večanja vrednosti kapitala kakor tudi vnovčenje blagovnega produkta, v katerem se prikaže povečana vrednost kapitala, se torej sklene z \( \KPEB' \KPEcrta \KPED' \).

Predpostavili smo torej enostavno reprodukcijo, to se pravi, da se \( \KPEd \KPEcrta \KPEb \) popolnoma loči od \( \KPED \KPEcrta \KPEB \). Ker pripadata obe cirkulaciji, tako \( \KPEb \KPEcrta \KPEd \KPEcrta \KPEb \) kakor tudi \( \KPEB \KPEcrta \KPED \KPEcrta \KPEB \), po svoji splošni obliki blagovni cirkulaciji (in zato tudi ne kažeta nobenih razlik v vrednosti med skrajnima točkama), je prav lahko videti, kakor vidi vulgarna ekonomija, v kapitalističnem produkcijskem procesu zgolj produkcijo blaga, uporabnih vrednosti za kakršno koli vrsto porabe, ki jih kapitalist producira edino zato, da bi jih nadomestil z blagom drugačne uporabne vrednosti ali da bi jih z njim zamenjal, kakor napačno trdi vulgarna ekonomija.

\( \KPEB' \) nastopa že od početka kot blagovni kapital; in namen vsega procesa, obogatitev (povečanje vrednosti), ne samo ne izključuje z velikostjo presežne vrednosti (torej tudi kapitala) naraščajoče konsumpcije kapitalista, ampak jo, nasprotno, vključuje.

V cirkulaciji kapitalistovega dohodka služi producirano blago \( \KPEb \) (oziroma njemu idealno ustrezajoči del blagovnega produkta \( \KPEB' \)) v resnici le zato, da ga zamenja najprej v denar, iz denarja pa v vrsto drugega blaga, ki služi zasebni potrošnji. Pri tem pa ne smemo spregledati drobne okoliščine, da je \( \KPEb \) blagovna vrednost, ki kapitalista ni nič stala, da je utelešenje presežnega dela, da torej stopa na oder izvirno\KPEstran kot sestavni del blagovnega kapitala \( \KPEB' \). Ta \( \KPEb \) sam je torej že po svojem obstoju vezan na krožni tok cirkulirajoče kapitalske vrednosti. Če se ta tok zaustavi ali sicer na kakršen koli način moti, se omeji ali povsem odpade ne le konsumpcija \( \KPEb \), ampak hkrati tudi prodaja blagovne vrste, ki tvori nadomestilo za \( \KPEb \). To se zgodi tudi tedaj, če \( \KPEB' \KPEcrta \KPED' \) spodleti ali če se proda le del \( \KPEB' \).

Videli smo, da pripada \( \KPEb \KPEcrta \KPEd \KPEcrta \KPEb \) kot cirkulacija kapitalistovega dohodka cirkulaciji kapitala samo dotlej, dokler je \( \KPEb \) del vrednosti \( \KPEB' \), kapitala v njegovi funkcijski obliki blagovnega kapitala; toda brž ko se z \( \KPEd \KPEcrta \KPEb \) osamosvoji, torej v vsej obliki \( \KPEb \KPEcrta \KPEd \KPEcrta \KPEb \), ne pripada gibanju kapitala, ki ga je založil kapitalist, čeprav izhaja iz tega kapitala. Cirkulacija dohodka je združena s cirkulacijo založenega kapitala toliko, kolikor predpostavlja obstoj kapitala obstoj kapitalista, obstoj kapitalista pa zahteva, da troši presežno vrednost.

V okviru splošne cirkulacije deluje \( \KPEB' \), na primer preja, le kot blago; kot moment cirkulacije kapitala pa deluje kot \emph{blagovni kapital}, kot podoba, ki jo kapitalska vrednost izmenoma privzema in odlaga. Po prodaji preje trgovcu je izločena iz procesa kroženja tistega kapitala, katerega produkt je, je pa kot blago kljub temu še naprej v toku splošne cirkulacije. Cirkulacija iste količine blaga se nadaljuje, čeprav ni nič več moment v samostojnem krožnem toku predilčevega kapitala. Dejanska dokončna metamorfoza količine blaga, ki jo vrže kapitalist v cirkulacijo, \( \KPEB \KPEcrta \KPED \), njen dokončni odhod v konsumpcijo, je torej lahko v času in prostoru docela ločena od metamorfoze, v kateri deluje ta količina blaga kot njegov blagovni kapital. Ista metamorfoza, ki je v cirkulaciji kapitala že opravljena, se mora na področju splošne cirkulacije še izvesti.

Stvar se prav nič ne spremeni, če vstopi preja znovič v krožni tok kakega drugega industrijskega kapitala. Splošna cirkulacija obsega prav tako prepletanje krožnih tokov različnih samostojnih delov družbenega kapitala, torej skupnost posameznih kapitalov, kakor tudi cirkulacijo vrednosti, ki ne \KPEstran pridejo na trg kot kapital oziroma ki vstopajo v individualno konsumpcijo.

Razmerje med krožnim tokom kapitala kot delom splošne cirkulacije in kot členom samostojnega krožnega toka se pokaže tudi, če opazujemo cirkulacijo \( \KPED' = \KPED + \KPEd \). \( \KPED \) kot denarni kapital nadaljuje krožni tok kapitala; \( \KPEd \) kot potrošnja dohodka (\( \KPEd \KPEcrta \KPEb \)) vstopi v splošno cirkulacijo, izpade pa iz krožnega toka kapitala. V ta krožni tok vstopi samo tisti del, ki deluje kot dodatni denarni kapital. V \( \KPEb \KPEcrta \KPEd \KPEcrta \KPEb \) deluje denar samo kot denar; smoter te cirkulacije je individualna kapitalistova konsumpcija. Značilno za bebavost vulgarne politične ekonomije je, da označuje to cirkulacijo, ki ne spada v krožni tok kapitala -- cirkulacijo kot dohodek potrošenega dela producirane vrednosti -- za karakteristični krožni tok kapitala.

V drugi fazi, \( \KPED \KPEcrta \KPEB \), se kapitalska vrednost \( \KPED = \KPEP \) (enako vrednosti produktivnega kapitala, ki prične tu krožni tok industrijskega kapitala) pojavi vnovič, brez presežne vrednosti, torej v isti velikosti vrednosti kakor v prvem stadiju krožnega toka denarnega kapitala \( \KPED \KPEcrta \KPEB \). Kljub različnemu mestu ima denarni kapital, v katerega se je sedaj spremenil blagovni kapital, isto funkcijo: spremeniti se mora v \( \KPEPs \) in \( \KPEDs \), v produkcijska sredstva in delovno silo.

Hkrati z \( \KPEb \KPEcrta \KPEd \) je torej kapitalska vrednost v funkciji blagovnega kapitala \( \KPEB' \KPEcrta \KPED' \) prešla fazo \( \KPEB \KPEcrta \KPED \) in stopa sedaj v dopolnilno fazo \( \KPED \KPEcrta \KPEBrazcepDsPs \); njena celotna cirkulacija je torej \( \KPEB \KPEcrta \KPED \KPEcrta \KPEBrazcepDsPs \).

Prvič. Denarni kapital \( \KPED \) je nastopal v obliki I (krožni tok \( \KPED \KPEpike \KPED' \)) kot prvotna oblika, v kateri se zalaga kapitalska vrednost; tu nastopa že od početka kot del tiste denarne vsote, v katero se je spremenil blagovni kapital v prvi cirkulacijski fazi \( \KPEB' \KPEcrta \KPED' \), torej že od početka kot sprememba \( \KPEP \), produktivnega kapitala, v denarno obliko, ki jo omogoča prodaja blagovnega produkta. Denarni kapital ni tu že od početka niti kot prvotna niti kot končna oblika kapitalske \KPEstran vrednosti, ker se lahko izvede faza \( \KPED \KPEcrta \KPEB \), ki dokonča fazo \( \KPEB \KPEcrta \KPED \), edinole tako, da znova odvrže denarno obliko. Zato tudi tisti del \( \KPED \KPEcrta \KPEB \), ki je hkrati \( \KPED \KPEcrta \KPEDs \), ni več zgolj založitev denarja, izvedena z nakupom delovne sile, ampak je založitev denarja, s katero dobi delovna sila v denarni obliki predujem taistih 1000 funtov preje, vrednih 50~f.~št., ki tvorijo del blagovne vrednosti, katero je ustvarila delovna sila. Denar, ki ga dobiva tu delavec v predujem, je le spremenjena ekvivalentna oblika dela vrednosti tiste blagovne vrednosti, ki jo je sam produciral. In že zaradi tega torej akt \( \KPED \KPEcrta \KPEB \), kolikor je \( \KPED \KPEcrta \KPEDs \), nikakor ni samo nadomestitev blaga v denarni obliki z blagom v uporabni obliki, ampak vključuje druge elemente, ki so neodvisni od splošne blagovne cirkulacije kot take.

\( \KPED' \) nastopa kot spremenjena oblika \( \KPED' \), ki je sam produkt preteklega delovanja \( \KPEP \), produkcijskega procesa; celotna denarna vsota \( \KPED' \) nastopa zato kot denarni izraz preleklega dela. V našem primeru je 10.000 funtov preje = 500~f.~št., produkt predilnega procesa; od tega 7440 funtov preje = založenemu konstantnemu kapitalu \( \KPEc \) = 372~f.~št.; 1000 funtov preje = založenemu variabilnemu kapitalu \( \KPEv \) = 50~f.~št.; in 1560 funtov preje = presežni vrednosti \( \KPEpv \) = 78~f.~št. Če se pri sicer nespremenjenih okoliščinah na novo založi od \( \KPED' \) samo začetni kapital, ki je enak 422~f.~št., dobi delavec prihodnji teden kot predujem v \( \KPED \KPEcrta \KPEDs \) samo del ta teden produciranih 10.000 funtov preje (denarno vrednost 1000 funtov preje). Kot rezultat \( \KPEB \KPEcrta \KPED \) je denar vedno izraz preteklega dela. Če se izvede dopolnilni akt \( \KPED \KPEcrta \KPEB \) takoj na blagovnem trgu, če se torej zamenja \( \KPED \) za obstoječe blago, ki je na trgu, pomeni spet spremembo preleklega dela iz ene oblike (denar) v drugo obliko (blago). \( \KPED \KPEcrta \KPEB \) pa se časovno razlikuje od \( \KPEB \KPEcrta \KPED \). Izjemoma je lahko istočasen, če na primer kapitalist, ki opravlja \( \KPED \KPEcrta \KPEB \), in kapitalist, za katerega je ta akt \( \KPEB \KPEcrta \KPED \), istočasno dajeta drug drugemu svoje blago, \( \KPED \) pa potem samo izravna saldo. Časovna razlika med izvedbo \( \KPEB \KPEcrta \KPED \) in izvedbo \( \KPED \KPEcrta \KPEB \) je lahko večja ali manjša. Čeprav predstavlja \( \KPED \) kot rezultat akta \( \KPEB \KPEcrta \KPED \) preteklo delo, lahko predstavlja \( \KPED \) za akt \( \KPED \KPEcrta \KPEB \) spremenjeno \KPEstran obliko blaga, ki ga sploh še ni na trgu, ampak bo prišlo nanj šele v prihodnosti, ker je treba iz vesti \( \KPED \KPEcrta \KPEB \) šele potem, ko se \( \KPEB \) vnovič producira. Prav tako lahko predstavlja \( \KPED \) blago, ki se producira istočasno z \( \KPEB \), katerega izraz je. V zamenjavi \( \KPED \KPEcrta \KPEB \) (nakup produkcijskih sredstev) lahko na primer nakupimo premog, preden ga v jami nakopljejo. Če predstavlja \( \KPEd \) denarno akumulacijo, če se ne potroši kot dohodek, lahko pomeni bombaž, ki bo pridelan šele naslednje leto. Isto velja pri trošenju kapitalistovega dohodka, \( \KPEd \KPEcrta \KPEb \). Prav tako za mezdo \( \KPEDs \) = 50~f.~št.; ta denar ni le denarna oblika preteklega dela delavcev, ampak hkrati nakazilo na istočasno ali prihodnje delo, ki se šele realizira ali se bo realiziralo v prihodnosti. Delavec lahko kupi z njim suknjo, ki bo narejena šele naslednji teden. To velja zlasti za zelo veliko število nujnih življenjskih potrebščin, ki jih je treba konsumirati skoraj že takoj, ko so producirane, da se ne bi pokvarile. Tako dobi delavec v denarju, v katerem mu izplačajo njegovo mezdo, spremenjeno obliko svojega lastnega prihodnjega dela ali dela drugih delavcev. Z enim delom njegovega preteklega dela mu d\'a kapitalist nakazilo na njegovo lastno prihodnje delo. Njegovo preteklo delo mu plača z njegovim lastnim sočasnim ali prihodnjim delom, ki tvori še neobstoječo zalogo. Tu popolnoma izgine predstava o ustvarjanju zaloge.

Drugič. V cirkulaciji \( \KPEB \KPEcrta \KPED \KPEcrta \KPEBrazcepDsPs \) menja isti denar dvakrat svoje mesto; kapitalist ga prejme najprej kot prodajalec in ga odda kot kupec; sprememba blaga v denarno obliko ima le ta namen, da ga iz denarne oblike spremeni spet v blagovno; denarna oblika kapitala, njegov obstoj kot denarnega kapitala, je torej v tem gibanju samo prehoden moment; z drugimi besedami, dokler se gibanje ne prekine, je denarni kapital le cirkulacijsko sredstvo, če služi za kupno sredstvo; kot pravo plačilno sredstvo služi, če kupujejo kapitalisti med seboj drug od drugega, kadar je torej potrebno le izravnati nepobotane terjatve.

Tretjič. Funkcija \KPEstran denarnega kapitala, pa najsi služi le za cirkulacijsko ali pa za plačilno sredstvo, posreduje, da nadomesti \( \KPEB \) z \( \KPEDs \) in \( \KPEPs \), se pravi, da se nadomesti preja, blagovni produkt, ki predstavlja rezultat produktivnega kapitala (po odbitku presežne vrednosti, ki se bo uporabila kot dohodek) z njegovimi produkcijskimi elementi, da se torej spremeni vrednost kapitala iz svoje blagovne oblike nazaj v tvorne elemente tega blaga; omogoča torej končno le povratno spremembo blagovnega kapitala v produktivni kapital.

Da bi potekal krožni tok normalno, je treba prodati \( \KPEB' \) po njegovi vrednosti in v celoti. Razen tega ne pomeni \( \KPEB \KPEcrta \KPED \KPEcrta \KPEB \) le nadomestitve enega blaga z drugim, ampak nadomestitev v istih vrednostnih razmerjih. Naša predpostavka je, da se to godi. V resnici pa se vrednosti produkcijskih sredstev spreminjajo; ravno kapitalistični produkciji je lastno neprestano spreminjanje vrednostnih razmerij že radi stalnega spreminjanja produktivnosti dela, ki je značilno za kapitalistično produkcijo. Na to spreminjanje vrednosti produkcijskih faktorjev, ki ga moramo kasneje razložiti, tu le opozarjamo. Spreminjanje produkcijskih ementov v blagovni produkt, iz \( \KPEP \) v \( \KPEB' \), se dogaja na področju produkcije, povratno spreminjanje iz \( \KPEB' \) v \( \KPEP \) na področju cirkulacije. Posreduje ga enostavna metamorfoza blaga. Njegova vsebina pa je moment reprodukcijskega procesa, gledanega kot celota. \( \KPEB \KPEcrta \KPED \KPEcrta \KPEB \) vključuje kot oblika cirkulacije kapitala funkcionalno določeno menjavo tvarine. Cirkulacija \( \KPEB \KPEcrta \KPED \KPEcrta \KPEB \) razen tega omogoča, da je \( \KPEB \) enak produkcijskim elementom blagovne količine \( \KPEB' \) in da ohranijo ti svoja prvotna medsebojna vrednostna razmerja; predpostavljamo torej ne le, da se blago kupuje in prodaja po svoji vrednosti, ampak tudi, da ne pretrpi med kroženjem nobene spremembe vrednosti. Drugače proces ne more potekati normalno.

V \( \KPED \KPEpike \KPED' \) je \( \KPED \) prvotna oblika kapitalske vrednosti, ki jo odloži, da bi si jo ponovno nadela. V \( \KPEP \KPEpike \KPEB' \KPEcrta \KPED' \KPEcrta \KPEB \KPEpike \KPEP \) je \( \KPED \) oblika, ki si jo nadene samo v procesu in ki jo že v njem spet odloži. Denarna oblika je tu le prehodna \KPEstran samostojna oblika vrednosti kapitala; kot \( \KPEB' \) je kapital prav tako željan, da bi si jo nadel, kakor je kot \( \KPED \) željan, da bi jo odložil, brž ko se zaprede vanjo, in se spet pretvoril v obliko produktivnega kapitala. Dokler vztraja v denarni podobi, ne deluje kot kapital in zato ne poveča svoje vrednosti; kapital leži neizkoriščen. \( \KPED \) služi tu kot cirkulacijsko sredstvo, vendar kot cirkulacijsko sredstvo kapitala. Videz samostojnosti, ki jo ima denarna oblika vrednosti kapitala v prvi obliki svojega krožnega toka (denarnega kapitala), izgine v tej drugi obliki, ki potemtakem preveri prvo obliko in jo reducira zgolj na posebno obliko. Če zadene druga metamorfoza \( \KPED \KPEcrta \KPEB \) na ovire (na trgu na primer ni produkcijskih sredstev), se kroženje, tok reprodukcijskega procesa, prekine kakor tedaj, kadar obleži kapital v obliki blagovnega kapitala. Razlika pa je tale: v denarni obliki lahko vzdrži dalj časa kakor v minljivi blagovni obliki. Ne preneha biti denar, kadar funkcionira kot denarni kapital; preneha pa biti blago in uporabna vrednost sploh, če se predolgo zadržuje v svoji funkciji blagovnega kapitala. Razen tega je v denarni obliki sposoben prevzeti namesto svoje prvotne oblike produkcijskega kapitala kakšno drugo, medtem ko kot \( \KPEB' \) sploh ne more nikamor.

\( \KPEB' \KPEcrta \KPED' \KPEcrta \KPEB \) vključuje samo za \( \KPEB' \), ustrezno njegovi obliki, cirkulacijske akte, ki so momenti njegove reprodukcije; za izvedbo \( \KPEB' \KPEcrta \KPED' \KPEcrta \KPEB \) pa je potrebna dejanska reprodukcija \( \KPEB \), v katerega se pretvori \( \KPEB' \); ta pa je odvisna od reprodukcijskih procesov zunaj reprodukcijskega procesa tistega individualnega kapitala, ki ga predstavlja \( \KPEB' \).

V obliki I pripravi \( \KPED \KPEcrta \KPEBrazcepDsPs \) le prvo spremembo denarnega kapitala v produktivni kapital; v obliki II pripravi povratno spremembo iz blagovnega kapitala v produktivnega; to je, dokler ostane naložba industrijskega kapitala ista, povratno spremembo blagovnega kapitala nazaj v iste produkcijske elemente, iz katerih je izšel. Tu se torej ne pojavlja, kakor v obliki I, kot pripravljalna faza produkcijskega procesa, ampak kot vrnitev k njemu, kot njegova obnovitev, torej \KPEstran kot predhodnik reprodukcijskega procesa in torej tudi ponovitve procesa večanja vrednosti.

Spet je treba pripomniti, da \( \KPED \KPEcrta \KPEB \) ni enostavna blagovna cirkulacija, ampak nakup blaga \( \KPEDs \), ki mora služiti produkciji presežne vrednosti, podobno kakor je \( \KPED \KPEcrta \KPEPs \) samo procedura, ki je za izvedbo tega smotra stvarno neizogibno potrebna.

Z izvedbo \( \KPED \KPEcrta \KPEBrazcepDsPs \) je spremenjen \( \KPED \) spet v produktivni kapital, v \( \KPEP \), in krožni tok se prične znova.

Razširjena oblika za \( \KPEP \KPEpike \KPEB' \KPEcrta \KPED' \KPEcrta \KPEB \KPEpike \KPEP \) je torej:

\[
    \KPEP
    \KPEpike
    \KPEB'
    \left\lbrace
    \begin{array}{c}
        \KPEB \\
        + \\
        \KPEb \\
    \end{array}
    \right\rbrace
    \begin{array}{c}
        \KPEcrta \\
         \\
        \KPEcrta \\
    \end{array}
    \left\lbrace
    \begin{array}{c}
        \KPED \\
        + \\
        \KPEd \\
    \end{array}
    \right\rbrace
    \begin{array}{c l c c}
        \KPEcrta & \KPEBrazcepDsPs & \KPEpike & \KPEP \\
         & & & \\
        \KPEcrta & \KPEb & & \\
    \end{array}
\]

Sprememba denarnega kapitala v produktivni kapital je nakup blaga zaradi produkcije blaga. Potrošnja spada v krožni tok kapitala samo, kolikor je produktivna; pogoj je, da s tako konsumiranim blagom producira presežno vrednost. To pa je nekaj čisto drugega kakor produkcija in tudi blagovna produkcija, katere smoter je obstoj producenta; menjava blaga za blago zaradi produkcije presežne vrednosti je nekaj popolnoma drugega kakor menjava produktov sama po sebi -- ki jo denar le posreduje. Ekonomisti pa vidijo v njej le to poslednje in dokazujejo, da hiperprodukcija ni mogoča.

Razen produktivne konsumpcije \( \KPED \), ki se spremeni v \( \KPEDs \) in \( \KPEPs \), vsebuje krožni tok prvi člen \( \KPED \KPEcrta \KPEDs \), ki je za delavca \( \KPEDs \KPEcrta \KPED = \KPEB \KPEcrta \KPED \). Od delavčeve cirkulacije \( \KPEDs \KPEcrta \KPED \KPEcrta \KPEB \), ki vključuje njegovo konsumpcijo, spada le prvi člen kot rezultat \( \KPED \KPEcrta \KPEDs \) v krožni tok kapitala. Drugi akt, namreč \( \KPED \KPEcrta \KPEB \), ne spada v kroženje individualnega kapitala, čeprav izhaja iz njega. Za razred kapitalistov pa je nujen trajni obstoj delavskega razreda, torej tudi delavčeva konsumpcija, ki jo posreduje \( \KPED \KPEcrta \KPEB \).

Akt \( \KPEB' \KPEcrta \KPED' \) zahteva tako za nadaljevanje krožnega toka kapitalske vrednosti kakor tudi za kapitalistovo konsumpcijo presežne vrednosti samo, da se \( \KPEB' \) spremeni v denar, proda.\KPEstran Seveda ga kupijo samo, ker je predmet uporabna vrednost, sposoben torej za konsumpcijo kakršne koli vrste, produktivne ali individualne. Če pa \( \KPEB' \) cirkulira dalje, na primer v rokah trgovca, ki je kupil prejo, nima to neposredno prav nobenega pomena za nadaljevanje krožnega toka individualnega kapitala, ki je produciral prejo in jo prodal trgovcu. Ves proces se neprekinjeno nadaljuje, z njim pa tudi individualna konsumpcija kapitalista in delavca, ki je od tega odvisna. To je važna točka za proučevanje kriz.

Kakor hitro se namreč \( \KPEB' \) proda, spremeni v denar, se lahko spremeni nazaj v materialne faktorje delovnega procesa in zato reprodukcijskega procesa. Neposredno je torej popolnoma nepomembno, ali kupi \( \KPEB' \) končni potrošnik ali pa trgovec, ki ga nato spet proda. Obseg blagovnih količin, ki jih ustvarja kapitalistična produkcija, se odloča z obsegom te produkcije in z njeno potrebo po nenehnem širjenju, nikakor pa ne z vnaprej določenim krogom povpraševanja in ponudbe, s potrebami, ki jih je treba zadovoljiti. Razen drugih industrijskih kapitalistov je lahko neposredni kupec množične produkcije edino še veletrgovec. Znotraj določenih mej lahko poteka reprodukcijski proces v istem ali pa v razširjenem obsegu, čeprav iz njega izšlo blago dejansko ni prišlo v individualno ali produktivno konsumpcijo. Konsumpcija blaga ni vključena v krožni tok kapitala, iz katerega je izšlo. Kakor hitro se na primer preja proda, se lahko prične krožni tok v preji predstavljene vrednosti kapitala znova, ne glede na to, kakšna je nadaljnja usoda prodane preje. Dokler se produkt prodaja, gre z vidika kapitalističnega producenta vse svojo redno pot. Kroženje kapitalske vrednosti, ki jo reprezentira, se ne prekine. In če se ta proces razširi -- kar vključuje razširjeno produktivno potrošnjo produkcijskih sredstev -- lahko spremlja to reprodukcijo kapitala razširjena individualna potrošnja (torej povpraševanje) delavcev, ker ga uvaja in posreduje produktivna potrošnja. Tako lahko naraste produkcija presežne vrednosti in z njo tudi individualna potrošnja kapitalista, celotni reprodukcijski proces je lahko v \KPEstran največjem razcvetu, vendar je kljub temu velik del blaga prišel v potrošnjo le navidezno, v resnici pa leži neprodano v skladiščih preprodajalcev, je torej še vedno na trgu. Zdaj meče val za valom blaga na trg, nazadnje pa se pokaže, da je prejšnji val pogoltala konsumpcija le navidezno. Blagovni kapitali drug drugemu odžirajo prostor na trgu. Da bi lahko prodali, prodajajo tisti, ki so prišli zadnji, pod ceno. Prejšnji valovi še niso razprodani, ko zapadejo plačilni roki zanje. Njihovi lastniki se morajo proglasiti za plačilno nesposobne ali pa prodati po vsaki ceni, da bi lahko plačali. Ta prodaja nima prav nobene zveze z dejanskim stanjem povpraševanja. Njen vzrok je le \emph{povpraševanje po plačilu}, absolutna nujnost spremeniti blago v denar. Tedaj izbruhne kriza. Ne pokaže se v neposrednem zmanjšanju konsumnega povpraševanja, povpraševanja za individualno konsumpcijo, ampak v zmanjšanju menjave kapitala za kapital, reprodukcijskega procesa kapitala.

Če je treba blago \( \KPEPs \) in \( \KPEDs \), v katero se je spremenil \( \KPED \), da bi opravil svojo funkcijo kot denarni kapital, kot kapitalska vrednost, ki je določena, da se spremeni nazaj v produktivni kapital -- če je treba to blago kupiti ali plačati ob različnih rokih, če tvori torej \( \KPED \KPEcrta \KPEB \) vrsto drug za drugim sledečih si nakupov in plačil, opravlja del \( \KPED \) akt \( \KPED \KPEcrta \KPEB \), medtem ko ostane drugi del v denarni obliki, da bo služil istočasnim ali zaporednim aktom \( \KPED \KPEcrta \KPEB \) šele takrat, ko bodo zahtevali pogoji procesa samega. Cirkulaciji se odtegne samo začasno, da bi v določenem trenutku stopil v akcijo, opravil svojo funkcijo. To kopičenje enega dela denarja je tako funkcija, ki jo določa njegova cirkulacija in je namenjena tej cirkulaciji. Njegov obstoj kot kupnega in plačilnega sklada, odložitev njegovega gibanja, stanje, ko je njegova cirkulacija prekinjena, je stanje, v katerem opravlja denar eno od svojih funkcij kot denarni kapital. Kot denarni kapital; kajti v tem primeru je denar, ki začasno miruje, sam del denarnega kapitala \( \KPED \) (\( \KPED' \) minus \( \KPEd = \KPED \)), tistega dela vrednosti blagovnega kapitala, ki = \( \KPEP \), vrednosti produktivnega kapitala, iz katerega izhaja kroženje. Po drugi strani ima ves denar, ki je odtegnjen cirkulaciji, obliko \KPEstran zaklada. Zakladna oblika denarja postane torej tu funkcija denarnega kapitala, tako kakor se v \( \KPED \KPEcrta \KPEB \) spremeni funkcija denarja kot kupnega ali plačilnega sredstva v funkcijo denarnega kapitala. To zaradi tega, ker ima kapitalska vrednost tu denarno obliko, ker je tu denarno stanje industrijskega kapitala v enem njegovih stadijev, ki ga zahteva povezanost krožnega toka. Obenem pa to znova potrjuje, da ne opravlja denarni kapital v kroženju industrijskega kapitala nobenih drugih funkcij kot denarne in da imajo te denarne funkcije le zaradi svoje povezanosti z drugimi stadiji tega kroženja obenem značaj kapitalskih funkcij.

To, da se kaže \( \KPED' \) kot razmerje \( \KPEd \) proti \( \KPED \), kot razmerje kapitala, samo po sebi ni nobena funkcija denarnega kapitala, ampak blagovnega kapitala \( \KPEB' \), ki kot razmerje \( \KPEb \) proti \( \KPEB \) tudi sam ni nič drugega kot rezultat produkcijskega procesa, v njem opravljenega povečanja vrednosti kapitala.

Če trči nadaljnji cirkulacijski proces na zapreke, tako da mora \( \KPED' \) zaradi zunanjih okoliščin, položaja na trgu ipd., odložiti svojo funkcijo \( \KPED \KPEcrta \KPEB \) ter zaradi tega krajši ali daljši čas obtičati v svoji denarni podobi, imamo zopet opravka z denarjem v stanju zaklada, ki se pojavlja tudi v enostavni cirkulaciji blaga, kakor hitro zunanje okoliščine prekinejo prehod od \( \KPEB \KPEcrta \KPED \) v \( \KPED \KPEcrta \KPEB \). To je neprostovoljno tvorjenje zaklada. V našem primeru ima denar tako obliko neizkoriščenega, latentnega denarnega kapitala. Vendar pa za zdaj tega ne bomo podrobneje raziskovali.

V obeh primerih pa je zadrževanje denarnega kapitala v njegovi denarni podobi rezultat prekinitve gibanja, bodisi to smotrno ali nesmotrno, prostovoljno ali neprostovoljno, skladno z njegovimi funkcijami ali ne.

\section{Akumulacija in reprodukcija v razširjenem obsegu}

Ker sorazmerja, v katerih se produkcijski proces lahko razširi, niso samovoljna, ampak tehnično določena, more realizirana presežna vrednost, četudi je namenjena kapitalizaciji, narasti do obsega, v katerem lahko dejansko deluje kot \KPEstran dodatni kapital ali se vključi v kroženje delujoče kapilalske vrednosti, pogosto šele potem, ko ponovi več krožnih tokov (dotlej se mora torej pogosto kopičiti). Presežna vrednost torej otrpne v zaklad in postane v tej obliki latenten denarni kapital. Latenten zato, ker ne more delovati kot kapital, dokler vztraja v denarni obliki.\footnote{Izraz »latenten« je povzet po fizikalnem pojmu latentne toplote, ki ga je sedaj teorija o spremembi energije skoraj popolnoma odpravila. Zaradi tega uporablja Marx v tretjem oddelku (kasnejša redakcija) namesto njega izraz, ki je povzet po pojmu potencialne energije: »potencialni« ali, po analogiji z D'Alembertovimi virtualnimi hitrostmi, »virtualni kapital«. -- F. E.} Tvorjenje zaklada se pokaže tako kot pojav, vključen v kapitalistični proces akumulacije, katerega spremlja, od katerega pa se hkrati bistveno razlikuje. Kajti s samim tvorjenjem latentnega denarnega kapitala se reprodukcijski proces ne razširja. Nasprotno. Latentni denarni kapital nastaja tu zato, ker kapitalistični producent ne more takoj razširiti obsega svoje produkcije. Če proda svoj presežni produkt producentu zlata ali srebra, ki meče novo zlato ali srebro v cirkulacijo, ali, kar je isto, trgovcu, ki uvaža iz tujine dodatno zlato ali srebro za del nacionalnega presežnega produkta, tedaj pomeni njegov latentni denarni kapital povečanje nacionalnega zlatega ali srebrnega zaklada. V vseh drugih primerih je na primer tistih 78~f.~št., ki so bili v rokah kupca cirkulacijsko sredstvo, privzelo v rokah kapitalista zgolj obliko zaklada; prišlo je torej le do drugačne porazdelitve nacionalnega zlatega ali srebrnega zaklada.

Če deluje denar v transakcijah našega kapitalista kot plačilno sredstvo (tako, da mora plačati kupec blago šele po krajšem ali daljšem roku), se presežna vrednost, ki je določena za kapitalizacijo, ne spremeni v denar, ampak v terjatve, v lastninski naslov na protivrednost, ki jo ima kupec mogoče že v posesti, mogoče pa jo šele pričakuje. Podobno kot denar, ki je naložen v obrestonosne papirje itd., ne gre v reprodukcijski proces krožnega toka, čeprav lahko vstopi v krožni tok drugih individualnih industrijskih kapitalov.

Kapitalistični \KPEstran produkciji daje v celoti njen značaj večanje vrednosti založene vrednosti kapitala, predvsem torej produkcija kolikor mogoče veliko presežne vrednosti, potem pa (glej l. knjigo, 22. poglavje) produkcija kapitala, torej spreminjanje presežne vrednosti v kapital. Akumulacija ali produkcija v razširjenem obsegu, ki je kot sredstvo za vedno obsežnejšo produkcijo presežne vrednosti, torej kot sredstvo za kapitalistovo bogatenje, njegov osebni smoter in vključena v splošno tendenco kapitalistične produkcije, pa postane v svojem nadaljnjem razvoju, kakor smo pokazali v prvi knjigi, neizogibna za vsakega posameznega kapitalista. Nenehno večanje njegovega kapitala postane pogoj, da ga ohrani. Vendar ni potrebno, da bi se vračali na tisto, kar smo obravnavali že poprej.

Najprej smo obravnavali enostavno reprodukcijo, pri kateri smo predpostavljali, da se vsa presežna vrednost potroši kot dohodek. V resnici se mora ob normalnih okoliščinah del presežne vrednosti vedno potrošiti kot dohodek, drug del pa kapitalizirati. Pri tem je prav vseeno, če se v določenih obdobjih producirana presežna vrednost kdaj docela potroši, drugič pa v celoti kapitalizira. V povprečju -- in splošni obrazec lahko ponazarja le povprečje -- se dogaja oboje. Da ne bi obrazca zapletli, pa je vendar pametneje, če vzamemo, da se celotna presežna vrednost akumulira. Obrazec \( \KPEP \KPEpike \KPEB' \KPEcrta \KPED' \KPEcrta \KPEBrazcepDsPsII \KPEpike \KPEP' \) izraža: produktivni kapital, ki se reproducira v večjem obsegu in v večji vrednosti in ki prične svoj drugi krožni tok ali, kar je isto, obnovi svoj prvi krožni tok kot povečani produktivni kapital. Kakor hitro se prične ta drugi krožni tok, je \( \KPEP \) spet izhodiščna točka; le da je \( \KPEP \) večji produktivni kapital, kakor je bil prvi \( \KPEP \). Tako je, če se začne v obrazcu \( \KPED \KPEpike \KPED' \) drugi krožni tok z \( \KPED' \), če deluje \( \KPED' \) kot \( \KPED \), kot založeni denarni kapital določene velikosti; ta denarni kapital je večji kot tisti, s katerim se je začel prvi krožni tok. Kakor hitro pa nastopi v funkciji založenega denarnega kapitala, izgine vsako znamenje o njegovem povečanju s kapitaliziranjem presežne vrednosti. Ta izvor je zbrisan v njegovi obliki denarnega \KPEstran kapitala, ki prične njegov krožni tok. Prav tako je s \( \KPEP' \), kakor hitro nastopi kot izhodišče novega krožnega toka.

Če primerjamo \( \KPEP \KPEpike \KPEP' \) z \( \KPED \KPEpike \KPED' \) ali s prvim krožnim tokom, ugotovimo, da nikakor nimata istega pomena. Vzet sam zase, kot posamezen krožni tok, izraža \( \KPED \KPEpike \KPED' \) le, da je \( \KPED \), denarni kapital (ali industrijski kapital v svojem krožnem toku kot denarni kapital), denar, ki rodi denar, vrednost, ki rodi vrednost, ustvarja presežno vrednost. V krožnem toku \( \KPEP \) pa je proces večanja vrednosti že opravljen, kakor hitro se konča prvi stadij, to je produkcijski proces; potem ko gresta skozi drugi stadij (prvi stadij cirkulacije) \( \KPEB' \KPEcrta \KPED' \), obstajata kapitalska in presežna vrednost že kot realizirani denarni kapital, kot \( \KPED' \), ki je zadnja skrajna točka v prvem krožnem toku. To, da je bila producirana presežna vrednost, je prikazano v obliki \( \KPEP \KPEpike \KPEP \), obravnavani v začetku (glej razširjeni obrazec na strani 83), z \( \KPEb \KPEcrta \KPEd \KPEcrta \KPEb \), ki v svojem drugem stadiju izpade iz cirkulacije kapitala in prikazuje cirkulacijo presežne vrednosti kot dohodka. Ta oblika, v kateri se ponazarja celotno gibanje s \( \KPEP \KPEpike \KPEP \), v kateri torej ni nobene vrednostne razlike med obema točkama, ponazarja večanje založene vrednosti, ustvarjanje presežne vrednosti, prav tako kot \( \KPED \KPEpike \KPED' \); samo da je akt \( \KPEB' \KPEcrta \KPED' \) v \( \KPED \KPEpike \KPED' \) zadnji stadij, v \( \KPEP \KPEpike \KPEP \) pa drugi stadij krožnega toka in prvi stadij cirkulacije.

V \( \KPEP \KPEpike \KPEP' \) ne izraža \( \KPEP' \), da se presežna vrednost producira, ampak da se producirana presežna vrednost kapitalizira, da je bil torej akumuliran kapital in da sestoji zato \( \KPEP' \) v nasprotju s \( \KPEP \) iz začetne vrednosti kapitala plus vrednosti kapitala, ki ga akumulira s svojim gibanjem.

\( \KPED' \) zgolj kot zaključek \( \KPED \KPEpike \KPED' \), enako kot tudi \( \KPEB' \), kakor nastopa v vseh teh krožnih tokih, ne izražata -- vzeta zase -- gibanja, ampak njegov rezultat: v blagovni ali denarni obliki realizirano povečanje vrednosti kapitala in s tem vrednost kapitala kot \( \KPED + \KPEd \) ali kot \( \KPEB + \KPEb \), kot razmerje kapitalske vrednosti nasproti njeni presežni vrednosti kot svoji potomki. Ta rezultat izražata kot različni cirkulacijski obliki povečane vrednosti kapitala. Toda niti v obliki \( \KPEB' \) niti v obliki \KPEstran \( \KPED' \) doseženo povečanje s\^amo ni funkcija bodisi denarnega, bodisi blagovnega kapitala. Kot posebni, različni obliki, kot načina obstoja, ki ustrezata posebnim funkcijam industrijskega kapitala, lahko opravljata denarni kapital samo denarne, blagovni kapital pa sam\'o blagovne funkcije; med seboj se razlikujeta le kot denar in blago. Prav tako lahko sestoji industrijski kapital v svoji obliki produktivnega kapitala samo iz istih elementov, iz katerih sestoji vsak drug delovni proces, ki ustvarja produkte: po eni strani iz stvarnih delovnih pogojev (produkcijskih sredstev), po drugi pa iz delovne sile, ki se produktivno (smotrno) udejstvuje. Tako kot lahko obstaja industrijski kapital na produkcijskem področju samo v sestavi, ki ustreza produkcijskemu procesu nasploh, torej tudi nekapitalističnemu produkcijskemu procesu, lahko obstaja na cirkulacijskem področju samo v obeh ustreznih mu oblikah blaga in denarja. Toda kakor se razkriva vsota produkcijskih elementov že od začetka kot produktivni kapital zaradi tega, ker je delovna sila tuja delovna sila, ki jo je kupil kapitalist od njenega lastnika prav tako, kakor je kupil od drugih lastnikov blaga svoja produkcijska sredstva; tako kakor nastopa zaradi tega tudi produkcijski proces sam kot produktivna funkcija industrijskega kapitala, nastopata tudi denar in blago kot cirkulacijski obliki istega industrijskega kapitala, zatorej tudi njuni funkciji kot njegovi cirkulacijski funkciji, ki funkcije produktivnega kapitala bodisi uvajata, bodisi izhajata iz njih. Samo ker sta obe obliki funkcij, ki jih mora opraviti industrijski kapital, v različnih stadijih svojega procesa krožnega toka povezani, sta denarna in blagovna funkcija tu hkrati funkciji denarnega in blagovnega kapitala. Zato je n\'apak izvajati posebne lastnosti in funkcije, ki označujejo denar kot denar in blago kot blago, iz njihovega kapitalskega značaja; in prav tako n\'apak je tudi izvajati lastnosti produktivnega kapitala iz načina njegovega obstoja v produkcijskih sredstvih.

Kakor hitro se \( \KPED' \) ali \( \KPEB' \) utrdita kot \( \KPED + \KPEd \), \( \KPEB + \KPEb \), se pravi kot razmerje vrednosti kapitala nasproti presežni vrednosti kot potomki prve, se izraža to razmerje v obeh, enkrat \KPEstran v denarni, drugič v blagovni obliki, kar pa za bistvo stvari ni pomembno. To razmerje ne izvira zato niti iz lastnosti in funkcij, ki pripadajo denarju kot takemu, niti iz onih, ki pripadajo blagu kot takemu. V obeh primerih se lastnost, ki označuje kapital, to, da je vrednost, ki ustvarja vrednost, izraža edinole kot rezultat. \( \KPEB' \) je vedno produkt delovanja \( \KPEP \), in \( \KPED' \) je vedno le oblika \( \KPEB' \), spremenjena v krožnem toku industrijskega kapitala. Kakor hitro realizirani denarni kapital znova začne svojo posebno funkcijo kot denarni kapital, preneha zato izražati kapitalsko razmerje, ki je vsebovano v \( \KPED' = \KPED + \KPEd \). Ko je \( \KPED \KPEpike \KPED' \) mimo in prične \( \KPED' \) znova krožni tok, ne nastopa več kot \( \KPED' \), ampak kot \( \KPED \), četudi bi se kapitalizirala vsa presežna vrednost, ki jo vsebuje \( \KPED' \). V našem primeru prične drugi krožni tok z denarnim kapitalom 500~f.~št., namesto, kakor prvi, s 422~f.~št. Denarni kapital, ki začne krožni tok, je za 78~f.~št. večji kot prej; to razliko vidimo, če primerjamo prvi krožni tok z drugim; v okviru enega samega krožnega toka pa primerjava ni možna. Kot denarni kapital založenih 500~f.~št., od katerih je bilo prej 78~f.~št. presežna vrednost, nima prav nič drugačne vloge kakor 500~f.~št., s katerimi začne kak drug kapitalist svoj prvi krožni tok. Prav tako je v krožnem toku produktivnega kapitala. Povečani \( \KPEP' \) nastopi pri ponovnem začetku kot \( \KPEP \), enako kakor \( \KPEP \) v enostavni reprodukciji \( \KPEP \KPEpike \KPEP \).

V stadiju \( \KPED' \KPEcrta \KPEBrazcepDsPsII \) se kaže povečani obseg le z \( \KPEB' \), ne pa z \( \KPEDs' \) in \( \KPEPs' \). Ker je \( \KPEB \) vsota \( \KPEDs \) in \( \KPEPs \), kaže že \( \KPEB' \), da je vsota v njem vsebovanih \( \KPEDs \) in \( \KPEPs \) večja kot prvotni \( \KPEP \). Razen tega pa bi bila označba \( \KPEDs' \) in \( \KPEPs' \) napačna, ker vemo, da je naraščanje kapitala povezano s spremembo njegove vrednostne sestave; s spreminjanjem sestave narašča vrednost \( \KPEPs \), vrednost \( \KPEDs \) pa se vedno zmanjšuje relativno, pogosto tudi absolutno.

\section{Denarna akumulacija}

Ali \KPEstran se \( \KPEd \), vnovčena presežna vrednost, takoj spet doda cirkulirajoči kapitalski vrednosti in lahko vstopi tako v proces krožnega toka skupno s kapitalom \( \KPED \) v obsegu \( \KPED' \), je odvisno od okoliščin, ki so neodvisne od samega obstoja \( \KPEd \). Če naj služi \( \KPEd \) kot denarni kapital v kakem drugem samostojnem podjetju, ustanovljenem poleg prvega, potem je gotovo, da ga je mogoče uporabiti za to samo, če ima za tako podjetje zahtevano minimalno velikost. Če naj se uporabi za razširitev prvotnega podjetja, zahtevajo razmerja tvarnih faktorjev in njihova vrednostna razmerja prav tako določeno minimalno velikost \( \KPEd \). Vsa v tem podjetju delujoča produkcijska sredstva niso med seboj samo v kakovostnem, ampak tudi v določenem količinskem razmerju, imajo določen sorazmeren obseg. Ta tvarna in na njih zasnovana vrednostna razmerja faktorjev, ki preidejo v produktivni kapital, določajo najmanjši obseg, ki ga mora imeti \( \KPEd \), da se lahko spremeni v dodatna produkcijska sredstva in delovno silo ali samo v prva kot prirastek produktivnega kapitala. Tako predilec ne more povečati števila svojih vreten, če ne nabavi hkrati ustreznih mikalnikov in strojev za debelo predenje, da ne omenimo povečanih izdatkov za bombaž in delovno silo, ki jih zahteva takšna razširitev podjetja. Da ga je mogoče razširiti, mora doseči presežna vrednost že precejšnjo vsoto (po navadi računajo po 1~f.~št. novih nabav na vreteno). Dokler nima \( \KPEd \) tega najmanjšega obsega, se mora krožni tok kapitala večkrat ponoviti, da lahko deluje vsota vseh \( \KPEd \), ki jih postopoma ustvarja, skupno z \( \KPED \), torej v \( \KPED' \KPEcrta \KPEBrazcepDsPsII \). Že same spremembe v podrobnostih, na primer pri predilnih strojih, zahtevajo večje izdatke za predilni material, pomnožitev strojev za debelo predenje itd., če jih napravijo produktivnejše. Medtem se \( \KPEd \) torej kopiči; njegovo kopičenje pa ni njegova lastna funkcija, ampak rezultat ponavljanih \( \KPEP \KPEpike \KPEP \). Njegova funkcija je, da vztraja v denarnem stanju, dokler se iz ponavljanih krožnih tokov večanja \KPEstran vrednosti, torej od zunaj, ne poveča toliko, da doseže minimalno velikost, ki jo zahteva njegova aktivna funkcija. Edino v tej velikosti se lahko vključi kot denarni kapital, v danem primeru kot akumulirani del delujočega denarnega kapitala \( \KPED \), v njegovo delovanje. Dotlej pa se kopiči in obstaja le v obliki nastajajočega, naraščajočega zaklada. Potemtakem je denarna akumulacija, tvorjenje zaklada, tu proces, ki začasno spremlja dejansko akumulacijo, razširitev obsega, v katerem deluje industrijski kapital. Začasno, kajti dokler ostane zaklad v stanju zaklada, ne deluje kot kapital, se ne udeležuje procesa večanja vrednosti, ostaja denarna vsota, ki narašča samo zaradi tega, ker leti denar, nastal brez njenega sodelovanja, v isto skrinjo.

Oblika zaklada je zgolj oblika denarja, ki ne cirkulira, denarja, katerega cirkulacija je bila pretrgana in se zaradi tega shranjuje v svoji denarni obliki. Kar zadeva sam proces tvorjenja zaklada, ga nahajamo v vsaki blagovni produkciji, kot namen samemu sebi pa le v njenih nerazvitih predkapitalističnih oblikah. V obravnavanem primeru pa nastopa zaklad kot oblika denarnega kapitala in tvorjenje zaklada kot proces, ki prehodno spremlja akumulacijo kapitala, ker in kolikor se pojavlja denar kot \emph{latentni denarni kapital}; ker je tvorjenje zaklada, zakladno stanje presežne vrednosti, ki ima denarno obliko, funkcionalno določen, zunaj krožnega toka kapitala potekajoči pripravljalni stadij za spremembo presežne vrednosti v resnično delujoči kapital. Denarni kapital je torej latenten zaradi te svoje določenosti. Zaradi tega določa vsakokratna vrednostna sestava produktivnega kapitala tudi obseg, ki ga mora doseči za vstop v proces. Dokler pa vztraja v zakladnem stanju, še ne deluje kot denarni kapital, je še neizkoriščeni denarni kapital; ne v svojem delovanju prekinjeni, kot prej, ampak za svoje delovanje še nesposobni denarni kapital.

Tu obravnavamo kopičenje denarja v njegovi prvotni, realni obliki, v kateri je dejanski denarni zaklad, lahko pa ima tudi obliko zahtevkov na plačilo, terjatev kapitalista, ki \KPEstran je prodal \( \KPEB' \). Druge oblike, v katerih obstaja ta latentni kapital v vmesnem razdobju kot denar, ki rodi denar, na primer kot obrestujoča se vloga v banki, v menicah ali raznovrstnih vrednostnih papirjih, pa ne spadajo sem. V denarju realizirana presežna vrednost opravlja potem posebne kapitalske funkcije zunaj krožnega toka industrijskega kapitala, iz katerega izvira; funkcije, ki, prvič, s tem krožnim tokom nimajo nobenega opravka, in, drugič, predpostavljajo kapitalske funkcije, ki so različne od funkcij industrijskega kapitala in ki jih tu še nismo raziskali.

\section{Rezervni sklad}

V pravkar obravnavani zakladni obliki, v kateri obstaja presežna vrednost, je zaklad, akumulacijski denarni sklad, denarna oblika, ki jo prehodno privzema akumulacija kapitala in ki je zaradi tega tudi njen pogoj. Ta akumulacijski sklad pa lahko opravlja tudi posebne postranske posle, se pravi, lahko vstopi v proces krožnega toka kapitala, četudi ta nima oblike \( \KPEP \KPEpike \KPEP' \), čeprav torej kapitalistična reprodukcija ni razširjena.

Če se podaljša proces \( \KPEB' \KPEcrta \KPED' \) čez svojo normalno mero, če se torej blagovni kapital nenormalno dolgo zadržuje v svojem spreminjanju v denarno obliko ali če se je po tej spremembi zvišala na primer cena produkcijskih sredstev, v katera se mora pretvoriti denarni kapital, nad raven ob začetku krožnega toka, se lahko uporabi zaklad, ki služi kot akumulacijski sklad, za to, da zavzame mesto denarnega kapitala ali njegovega dela. Denarni akumulacijski sklad služi tako kot rezervni sklad za izravnavanje motenj krožnega toka.

Kot tak rezervni sklad se razlikuje od sklada kupnih in plačilnih sredstev, ki smo ga obravnavali v krožnem toku \( \KPEP \KPEpike \KPEP \). Ta sredstva so del delujočega denarnega kapitala (torej obliki, v katerih sploh obstaja del v proces vključene kapitalske vrednosti), katerega deli pričenjajo delovati drug za drugim le v različnih časovnih obdobjih. V nepretrganem toku produkcijskega procesa se stalno ustvarja rezervni \KPEstran denarni kapital, ker pritečejo vplačila danes, izplačila pa zapadejo šele ob kasnejšem roku, ker se danes prodajo velike količine blaga, spet kupiti pa jih je treba šele čez nekaj dni; v teh premorih leži zato del cirkulirajočega kapitala vedno v denarni obliki. V nasprotju s tem pa rezervni sklad ni sestavni del delujočega kapitala, natančneje rečeno, denarnega kapitala, ampak kapitala, še preden se je akumuliral, presežne vrednosti, ki se še ni spremenila v aktivni kapital. Sicer pa se razume samo po sebi, da je v stiski kapitalistu prav malo mar, kakšne funkcije ima denar, katerega ima v rokah, ampak uporabi, kar ima, da bi ohranil v teku proces krožnega toka svojega kapitala. V našem primeru imamo na primer \( \KPED \) = 422~f.~št., \( \KPED' \) = 500~f.~št. Če leži del kapitala 422~f.~št. kot sklad plačilnih in kupnih sredstev, kot denarna zaloga, tedaj je preračunan tako, da pride pri nespremenjenih okoliščinah ves v krožni tok in da zanj tudi zadošča. Rezervni sklad pa je del 78~f.~št. presežne vrednosti; v proces krožnega toka kapitala 422~f.~št. vrednosti lahko vstopi le, če poteka ta v okoliščinah, ki se spreminjajo; je namreč del akumulacijskega sklada in nastopa tako, četudi se obseg reprodukcije ne poveča.

Akumulacijski denarni sklad pomeni že obstoj latentnega denarnega kapitala; torej spremembo denarja v denarni kapital.

Splošni obrazec krožnega toka produktivnega kapitala. ki vključuje enostavno in razširjeno reprodukcijo, je:

\[
\begin{array}{l}
    \phantom{---.} \thinspace \textrm{1} \phantom{--..} \thinspace \textrm{2} \\
    \begin{tikzpicture}[baseline=(current bounding box.center), remember picture]
        \matrix[matrix of math nodes, inner sep=0pt, column sep=0pt]{
          \KPEP & 
          \KPEpike & 
          \node (B1) {\KPEB'}; & 
          \KPEcrta & 
          \node (D1) {\KPED'}; & 
          \textrm{. } & 
          \node (D2) {\KPED}; & 
          \KPEcrta & 
          \node (B2) {\KPEB}; & 
          \KPErazcep^{\KPEDs}_{\KPEPs} & 
          \KPEpike & 
          \KPEP &
          \textrm{ (} & 
          \KPEP' & 
          \textrm{).}\\
        };
        \draw [overbrace style] (B1.north) -- (D1.north);
        \draw [overbrace style] (D2.north) -- (B2.north);
    \end{tikzpicture}
\end{array}
\]

Če je \( \KPEP = \KPEP \), potem je \( \KPED \) v 2 enak \( \KPED' \) minus \( \KPEd \); če je \( \KPEP = \KPEP' \), potem je \( \KPED \) v 2 večji kot \( \KPED' \) minus \( \KPEd \); to pomeni, da se je \( \KPEd \) v celoti ali deloma spremenil v denarni kapital.

Krožni tok produktivnega kapitala je oblika, v kateri obravnava klasična ekonomija proces krožnega toka industrijskega kapitala.

\end{document}
