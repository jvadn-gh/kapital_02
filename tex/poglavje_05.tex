\documentclass[kapital_02.tex]{subfiles}

\begin{document}

Kakor\footnote{Od tod naprej rokopis IV.}%TODO ta fusnota je v naslovu
\KPEstran\ smo videli, se giblje kapital prek produkcijskega področja in prek dveh faz cirkulacijskega področja v določenem časovnem zaporedju. Čas, v katerem se zadržuje na produkcijskem področju, je njegov produkcijski čas; čas, v katerem se zadržuje na cirkulacijskem področju, pa njegov cirkulacijski ali menjalni čas. Celotni čas, v katerem opiše svoj krožni tok, je torej enak vsoti produkcijskega in cirkulacijskega časa.

Produkcijski čas obsega seveda razdobje delovnega procesa; vendar ga ta ne izpolni v celoti.
Najprej moramo vedeti, da obstaja del konstantnega kapitala v delovnih sredstvih, kakor strojih, zgradbah itd., ki služijo do konca svojega življenja v istih, vedno znova ponavljajočih se delovnih procesih. 
Občasna prekinitev delovnega procesa, na primer ponoči, pretrga sicer delovanje teh delovnih sredstev, ne pa njihovega bivanja na produkcijskem mestu.
Temu pripadajo ne le medtem, ko delujejo, ampak tudi medtem, ko ne delujejo.
Na drugi strani mora imeti kapitalist pripravljeno določeno zalogo surovin in pomožnih snovi, da lahko poteka produkcijski proces v krajših ali daljših odsekih v vnaprej določenem obsegu, ne da bi bil zaradi naključij vsakodnevne ponudbe odvisen od trga. 
Ta zaloga surovin itd.\ se le postopoma produktivno konsumira.
Zato nastane razlika med njenim produkcijskim časom\footnote{Produkcijski čas razumemo tu aktivno: produkcijski čas produkcijskih sredstev ni čas, v katerem jih producirajo, ampak čas, v katerem sodelujejo v produkcijskem procesu kakega blagovnega produkta. — F.\ E.} 
in njenim\KPEstran\ delovalnim časom.
Produkcijski čas produkcijskih sredstev obsega torej na splošno: 1.\ čas, v katerem delujejo kot produkcijska sredstva, to je, služijo v produkcijskem procesu, 2.\ premore, v katerih je produkcijski proces, torej tudi delovanje njemu priključenih produkcijskih sredstev, pretrgan, 3.\ čas, v katerem so kot pogoji procesa sicer pripravljena, so torej že produktivni kapital, niso pa še vstopila v produkcijski proces.

Doslej obravnavana razlika je vsakokrat razlika med časom, v katerem se zadržuje produktivni kapital na produkcijskem področju, in časom, v katerem se zadržuje v produkcijskem procesu.
Produkcijski proces pa lahko sam zahteva prekinitve delovnega procesa in s tem delovnega časa, premore, v katerih prepušča delovni predmet delovanju fizikalnih procesov brez nadaljnjega sodelovanja človeškega dela.
V tem primeru se produkcijski proces, torej delovanje produkcijskih sredstev, nadaljuje, čeprav je delovni proces prekinjen, s tem pa tudi delovanje produkcijskih sredstev kot delovnih sredstev.
Tako npr.\ žito, ki je posejano, vino, ki vre v kleti, surovine v mnogih manulakturah, kakor na primer v strojarnah, kjer te surovine predelujejo v kemičnih procesih.
Produkcijski čas je tu daljši kakor delovni.
Razlika med obema je presežek produkcijskega časa nad delovnim.
Ta presežek temelji vedno na tem, da se produktivni kapital \emph{latentno} zadržuje na produkcijskem področju, ne da bi deloval v produkcijskem procesu samem, ali da deluje v produkcijskem procesu, ne da bi bil v delovnem.

Tisti del latentnega produktivnega kapitala, ki leži pripravljen samo kot pogoj produkcijskega procesa, kakor bombaž, premog itd.\ v predilnici, ne ustvarja niti produkta niti vrednosti.
Je nedejaven kapital, čeprav je njegova dejavnost pogoj za nepretrgani tek produkcijskega procesa.
Zgradbe, aparati itd., ki so potrebni kot shranjevalci produktivnih zalog (latentnega kapitala), so pogoji produkcijskega procesa in so zato sestavni deli založenega produktivnega kapitala.
Svojo funkcijo opravljajo kot varuhi produktivnih sestavnih delov v pripravljalnem stadiju. 
Če so\KPEstran\ v tem stadiju potrebni delovni procesi, podražujejo surovine itd., pomenijo pa produktivno delo in ustvarjajo presežno vrednost, ker del tega dela, tako kakor vsakega drugega mezdnega dela, ni plačan. 
Normalne prekinitve celotnega produkcijskega procesa, torej premori, ko produktivni kapital ne deluje, ne ustvarjajo niti vrednosti niti presežne vrednosti. 
Od tod prizadevanje, da se dela tudi ponoči (I.\ knjiga, 8.\ poglavje, 4).
 — Premori v delovnem času, prek katerih mora delovni predmet 
med produkcijskim procesom, ne ustvarjajo niti vrednosti niti presežne vrednosti; pospešujejo pa produkt, so del njegovega življenja, proces, skozi katerega mora.
Vrednost aparatov itd.\ se prenaša na produkt sorazmerno s celotnim časom, v katerem delujejo; delo samo postavlja produkt v ta stadij, uporaba teh aparatov pa je prav tako pogoj produkcije kakor razprašitev dela bombaža, ki ne preide v produkt, ki pa vendar prenese nanj svojo vrednost.
Drugi del latentnega kapitala, kakor zgradbe, stroji itd., to se pravi delovna sredstva, katerih funkcijo prekinjajo le redni premori produkcijskega procesa — neredne prekinitve zaradi omejitve produkcije, kriz itd.\ so čista izguba — dodaja vrednost, ne sodeluje pa pri ustvarjanju produkta; celotno vrednost, ki jo doda produktu, določa njegovo povprečno trajanje; vrednost izgublja zato, ker izgublja uporabno vrednost, tako takrat, kadar deluje, kakor tudi tedaj, ko ne deluje.

Naposled se vrednost konstantnega dela kapitala, ki ostaja v produkcijskem procesu, čeprav je delovni proces pretrgan, spet pojavi v rezultatu produkcijskega procesa.
Delo samo postavi tu produkcijska sredstva v takšne pogoje, v katerih opravijo sama od sebe določene naravne procese, katerih rezultat je določen koristni učinek ali spremenjena oblika uporabne vrednosti.
Delo vedno prenaša vrednost produkcijskih sredstev na produkt, če jih dejansko smotrno porabi kot produkcijska sredstva.
Glede tega se nič ne spremeni, če mora delo, da doseže ta učinek, z uporabo delovnih sredstev nepretrgano delovati na predmet dela, ali pa zadostuje že, da spravi delo produkcijska sredstva v stanje, v katerem sama od sebe, brez nadaljnjega\KPEstran\ sodelovanja dela, zaradi naravnih procesov doživijo nameravane spremembe.

Najsi je razlog za presežek produkcijskega časa nad delovnim takšen ali drugačen — bodisi da tvorijo produkcijska sredstva le latentni produktivni kapital, da so torej šele na pragu pravega produkcijskega procesa, bodisi da prekinjajo njihovo delovanje premori v produkcijskem procesu ali da zahteva produkcijski proces sam prekinitve delovnega procesa — v nobenem od teh primerov produkcijska sredstva ne vsrkujejo dela.
Če ne vsrkujejo dela, tudi presežnega dela ne vsrkujejo.
Dokler je vrednost produktivnega kapitala v tistem delu svojega produkcijskega časa, ki presega delovnega, se zato ne povečuje, naj je izvedba procesa povečanja vrednosti še tako neločljiva od teh njegovih premorov.
Jasno je, da sta produktivnost in povečanje vrednosti danega produktivnega kapitala v da nem obdobju tem večja, čim bolj se ujemata produkcijski in delovni čas.
Od tod težnja kapitalistične produkcije čimbolj skrajšati presežek produkcijskega časa nad delovnim.
Toda čeprav se lahko produkcijski čas kapitala razlikuje od njegovega delovnega časa, ga vedno obsega in presežek sam je pogoj produkcijskega procesa.
Produkcijski čas je torej vedno čas, v katerem producira kapital uporabne vrednosti in povečuje sam sebe, deluje torej kot produktivni kapital, čeprav vključuje čas, v katerem je ali latenten ali pa sicer producira, se pa ne povečuje.

Na cirkulacijskem področju biva kapital kot blagovni in denarni kapital.
Oba njegova cirkulacijska procesa obstajata v tem, da se spremeni iz blagovne oblike v denarno in iz denarne v blagovno.
Okoliščina, da je tu sprememba blaga v denar obenem realizacija v blagu utelešene presežne vrednosti in da je sprememba denarja v blago hkrati sprememba ali vrnitev kapitalske vrednosti v podobo njenih produkcijskih elementov, ničesar ne spremeni na tem, da so ti procesi kot cirkulacijski procesi procesi enostavne blagovne metamorfoze.

Cirkulacijski\KPEstran\ in produkcijski čas izključujeta drug drugega.
Med svojim cirkulacijskim časom kapital ne deluje kot produktivni kapital in zato ne producira niti blaga niti presežne vrednosti.
Če opazujemo krožni tok v njegovi najbolj enostavni obliki, tako da prestopi celotna vrednost kapitala na mah iz ene faze v drugo, je čisto jasno, da je produkcijski proces pretrgan in je zato tudi večanje vrednosti kapitala pretrgano, dokler traja cirkulacijski čas, in da se bo v skladu z njegovo dolgostjo produkcijski proces hitreje ali počasneje obnovil.
Če pa nasprotno pretekajo različni deli kapitala krožni tok drug za drugim, tako da se izvrši krožni tok celotne vrednosti kapitala kot zaporedje krožnih tokov različnih njegovih delov, pa je očitno, da mora biti tisti njegov del, ki stalno deluje na produkcijskem področju, toliko manjši, kolikor dlje se njegovi alikvotni deli trajno zadržujejo na cirkulacijskem področju.
Širjenje in krčenje cirkulacijskega časa deluje torej kot negativna meja krčenja in širjenja produkcijskega časa ali obsega, v katerem kapital določene velikosti deluje kot produktivni kapital.
Čim bolj so cirkulacijske metamorfoze kapitala samo idealne, se pravi, čim bolj je cirkulacijski čas enak ničli ali se bliža ničli, toliko bolj deluje kapital, toliko večja je njegova produktivnost in večanje vrednosti.
Če dela na primer kak kapitalist po naročilu, tako da dobi plačilo ob dobavi produkta, in to v obliki njegovih lastnih produkcijskih sredstev, se cirkulacijski čas približuje ničli.

Cirkulacijski čas kapitala torej na splošno omejuje njegov produkcijski čas in s tem proces večanja njegove vrednosti.
Omejuje pa ga sorazmerno s svojim lastnim trajanjem.
Ta pa se lahko na zelo različne načine daljša ali krči in torej tudi v zelo različni meri omejuje produkcijski čas kapitala.
Politična ekonomija pa vidi le \emph{zunanjost} pojava, namreč vpliv cirkulacijskega časa na proces večanja vrednosti kapitala nasploh.
Ta negativni vpliv šteje za pozitivnega, ker so pozitivne njegove posledice.
Te zunanjosti se oklepa toliko trdneje, ker navidezno dokazuje, da je v kapitalu mistični vir večanja vrednosti, ki mu priteka iz sfere cirkulacije popolnoma neodvisno od njegovega produkcijskega\KPEstran\ procesa in zato od izkoriščanja dela.
Kasneje bomo videli, da se pusti slepiti s tem videzom celo znanstvena ekonomija.
Kakor bomo prav tako pokazali, ga utrjuje več pojavov: 1.\ kapitalistični način preračunavanja profita, v katerem nastopa negativni razlog kot pozitivni, ker na kapitle na različnih naložbenih področjih, kjer je različen samo cirkulacijski čas, deluje daljši cirkulacijski čas kot razlog višanja cen, skratka, kot eden od razlogov za izravnavanje profitov.
2.\ Cirkulacijski čas je samo moment časa obrata; le-ta pa vključuje čas produkcije oziroma reprodukcije.
Kar je sad časa obrata, se zdi, da je sad cirkulacijskega časa.
3.\ Sprememba blaga v variabilni kapital (mezdo) je mogoča le, če se prej spremeni v denar.
Pri akumulaciji kapitala poteka torej sprememba v dodatni variabilni kapital na področju cirkulacije ali med cirkulacijskim časom.
Zato se zdi, da izhaja tako dosežena akumulacija iz cirkulacijskega časa.

Znotraj cirkulacijskega področja preteče kapital — bodisi v enem ali v drugem vrstnem redu — obe nasprotni si fazi \(\KPEB\KPEcrta\KPED\) in \(\KPED\KPEcrta\KPEB\).
Tudi njegov cirkulacijski čas se torej cepi na dva dela, na čas, ki ga potrebuje, da se spremeni iz blaga v denar, in na čas, ki ga potrebuje, da se spremeni iz denarja v blago.
Iz analize enostavne blagovne cirkula cije že vemo (1.\ knjiga, 3.\ poglavje), da je \(\KPEB\KPEcrta\KPED\), prodaja, najtežji del njegove metamorfoze in da zavzema zato v običajnih okoliščinah večji del cirkulacijskega časa.
Kot denar je vrednost v svoji vedno zamenljivi obliki.
Kot blago se mora prej spremeniti v denar, da pridobi to obliko neposredne zamenljivosti in s tem stalne delovne pripravljenosti.
Vendar pa gre pri cirkulacijskem procesu kapitala v njegovi fazi \(\KPED\KPEcrta\KPEB\) za njegovo spremembo v blago, ki tvori določene elemente produktivnega kapitala v dani naložbi.
Zgodi se lahko, da produkcijskih sredstev ni na trgu, ampak jih je treba šele producirati, ali da jih je treba dobiti z oddaljenih trgov, ali da se njihova običajna dobava pretrga, ali da se spremenijo cene itd., skratka, množica okoliščin, ki jih v enostavni menjavi oblike \(\KPED\KPEcrta\KPEB\) ni mogoče opaziti, ki pa vendar zahtevajo tudi za ta del cirkulacijske faze zdaj več, zdaj manj časa.
Kakor sta \(\KPEB\KPEcrta\KPED\)\KPEstran\ in \(\KPED\KPEcrta\KPEB\) ločena lahko časovno, sta lahko tudi prostorsko; nakupni in prodajni trg sta lahko prostorsko različna trga.
Pri tovarnah so na primer nakupovalci in prodajalci kaj pogosto različne osebe.
Pri blagovni produkciji je cirkulacija prav tako potrebna kakor produkcija sama, cirkulacijski agenti torej prav tako nujni kakor produkcijski.
Reprodukcijski proces vključuje obe funkciji kapitala, torej tudi nujnost, da ju kdo zastopa, bodisi kapitalist sam, bodisi mezdni delavec, njegov agent.
Zaradi tega pa nikakor ni treba cirkulacijskih agentov zamenjavati s produkcijskimi, kakor zaradi tega tudi ni treba zamenjavati funkcij blagovnega in denarnega kapitala s funkcijami produktivnega kapitala.
Cirkulacijske agente plačujejo produkcijski agenti.
Če pa kapitalisti, ki med seboj kupujejo in prodajajo, s tem aktom ne ustvarjajo niti produktov niti vrednosti, se to nič ne spremeni, če jim obseg njihovih kupčij omogoči in jih prisili, da prevale te funkcije na druge.
V mnogih podjetjih plačujejo nakupovalce in prodajalce z deležem dobička.
Fraza, da jih plačujejo konsumenti, ničesar ne pojasni.
Konsumenti lahko plačajo le, kolikor si kot produkcijski agenti sami producirajo ekvivalent v blagu ali ga pridobe od produkcijskih agentov, bodisi s kakšnim pravnim naslovom (kot njihovi družabniki itd.), bodisi z osebnimi storitvami.

Med \(\KPEB\KPEcrta\KPED\) in \(\KPED\KPEcrta\KPEB\)\ je razlika, ki nima nobenega opravka z raznoličnostjo oblik blaga in denarja, ampak izhaja iz kapitalističnega značaja produkcije.
Sama po sebi sta tako \(\KPEB\KPEcrta\KPED\) kakor \(\KPED\KPEcrta\KPEB\)\ zgolj prenosa določene vrednosti iz ene oblike v drugo.
\(\KPEB'\KPEcrta\KPED'\) pa je hkrati realizacija v \(\KPEB'\)vsebovane presežne vrednosti.
Ne tako tudi \(\KPED\KPEcrta\KPEB\).
Zatorej je prodaja važnejša kakor nakup.
V normalnih okoliščinah je akt \(\KPED\KPEcrta\KPEB\) potreben za povečanje vrednosti, ki jo izraža \(\KPED\), ne realizira pa presežne vrednosti; je uvod v njeno produkcijo, ne pa njeno dopolnilo.

Cirkulaciji blagovnega kapitala \(\KPEB'\KPEcrta\KPED'\) postavlja določene meje že oblika, ki jo ima blago, njegov obstoj kot uporabne vrednosti. 
Uporabne vrednosti so po naravi minljive.
Če v določenem času ne pridejo v produktivno ali osebno\KPEstran\ konsumpcijo, za kar so pač določene, če se, drugače povedano, v določenem času ne prodajo, se pokvarijo in
izgube s svojo uporabno vrednostjo tudi lastnost nosilca
menjalne vrednosti.
Kapitalska oziroma njej prirasla presežna vrednost, ki jo vsebujejo, gre po zlu.
Uporabne vrednosti ostanejo nosilke ohranjujoče se in povečujoče se vrednosti kapitala samo, dokler se nenehoma obnavljajo in reproducirajo, nadomeščajo z novimi uporabnimi vrednostmi iste ali druge vrste.
Njihova prodaja v njihovi končni blagovni obliki, tj. njihov vstop v produktivno ali osebno konsumpcijo, ki ga posreduje prodaja, pa je vedno znova pogoj njihove reprodukcije.
Če hočejo zaživeti v novi, morajo v določenem času zamenjati svojo staro uporabno obliko.
Menjalna vrednost se ohranja samo s takšnim stalnim obnavljanjem svojega telesa.
Uporabne vrednosti različnega blaga se pokvarijo hitreje ali počasneje; med njihovo produkcijo in njihovo porabo preteče lahko torej več ali manj časa; ne da bi propadle, lahko torej dalj ali manj časa vzdrže kot blagovni kapital v cirkulacijski fazi \(\KPEB\KPEcrta\KPED\), prenesejo kot blago krajši ali daljši cirkulacijski čas.
Meja cirkulacijskega časa blagovnega kapitala, ki jo postavlja pokvarljivost blagovnega telesa sama, je zgornja meja tega dela cirkulacijskega časa ali cirkulacijskega časa, v katerem lahko deluje blagovni kapital kot blagovni kapital.
Kolikor hitreje je kako blago minljivo, toliko hitreje po svoji produkciji se mora porabiti, torej tudi prodati, toliko manj se lahko oddalji od svojega produkcijskega mesta, toliko ožje je torej njegovo prostorsko cirkulacijsko območje, toliko bolj krajevnega značaja je njegov prodajni trg.
Čim hitreje mine torej blago, čim nižja je zaradi njegovih fizičnih lastnosti zgornja meja njegovega cirkulacijskega časa v podobi blaga, tem manj primerno je, da bi bilo predmet kapitalistične produkcije.
V to produkcijo se lahko vključi le v zelo obljudenih krajih ali kjer se zaradi razvoja prevoznih sredstev krčijo krajevne razdalje. 
Koncentracija produkcije kakega predmeta v malo rokah in v zelo obljudenih krajih pa lahko ustvari razmeroma velik trg tudi za take predmete; tako npr. za velike pivovarne, mlekarne ipd.

\end{document}