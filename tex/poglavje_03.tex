\documentclass[kapital_02.tex]{subfiles}

\begin{document}

Splošni \KPEstran obrazec za krožni tok blagovnega kapitala je:

\[
    \KPEB'
    \KPEcrta
    \KPED'
    \KPEcrta
    \KPEB
    \KPEpike
    \KPEP
    \KPEpike
    \KPEB'
\]

\( \KPEB' \) ni le produkt, ampak tudi predpostavka obeh prejšnjih krožnih tokov, ker je to, kar je za en kapital \( \KPED \KPEcrta \KPEB \), za drug kapital že \( \KPEB' \KPEcrta \KPED' \), če je vsaj del produkcijskih sredstev samih blagovni produkt drugih individualnih kapitalov, ki se gibljejo v svojem krožnem toku. V našem primeru so na primer premog, stroji itd. blagovni kapital lastnika premogovnika, kapitalističnega izdelovalca strojev, itd. Razen tega smo pokazali že v poglavju\ I,\ 4, da predpostavlja že prva ponovitev \( \KPED \KPEpike \KPED' \), še preden se dokonča ta drugi krožni tok denarnega kapitala, ne le krožni tok \( \KPEP \KPEpike \KPEP \), ampak tudi krožni tok \( \KPEB' \KPEpike \KPEB' \).

Če se reprodukcija razširi, je končni \( \KPEB' \) večji kakor začetni \( \KPEB' \) in ga bomo zaradi tega tu označevali z \( \KPEB'' \).

Razlika med tretjo obliko in prvima dvema je, prvič, v tem, da začne v tem primeru celotna cirkulacija krožni tok s svojima dvema nasprotujočima si fazama, medtem ko prekinja v obliki I cirkulacijo produkcijski proces, v obliki II pa je celotna cirkulacija s svojima dopolnjujočima se fazama le posredovalka reprodukcijskega procesa in tvori zato posredovalno gibanje med \( \KPEP \KPEpike \KPEP \). Pri \( \KPED \KPEpike \KPED' \) je oblika cirkulacije \( \KPED \KPEcrta \KPEB \KPEpike \KPEB' \KPEcrta \KPED' \textrm{ = } \KPED \KPEcrta \KPEB \KPEcrta \KPED' \). Pri \( \KPEP \KPEpike \KPEP \) je obratna \( \KPEB' \KPEcrta \KPED' \). \( \KPED \KPEcrta \KPEB \textrm{ = } \KPEB \KPEcrta \KPED \KPEcrta \KPEB \). V \( \KPEB' \KPEpike \KPEB' \) ima prav tako zadnje navedeno obliko.

Drugič. \KPEstran V ponovitvi krožnih tokov I in II, tudi če sta sklepni točki \( \KPED' \) in \( \KPEP' \) izhodišči obnovljenega krožnega toka, izgine oblika, v kateri sta nastali. \( \KPED' \textrm{ = } \KPED \textrm{ + } \KPEd \), \( \KPEP' \textrm{ = } \KPEP \textrm{ + } \KPEp \) začne novi proces spet kot \( \KPED \) in \( \KPEP \). V obliki III pa moramo označiti izhodišče \( \KPEB \) kot \( \KPEB' \), tudi pri obnovitvi krožnega toka v istem obsegu, in sicer iz naslednjega razloga. Kakor hitro začne v obliki I \( \KPED' \) kot tak nov krožni tok, deluje kot denarni kapital \( \KPED \), kot v denarni obliki založena kapitalska vrednost, ki naj se poveča. Količina založenega denarnega kapitala, narastla z akumulacijo iz prvega krožnega toka, se je povečala. Če znaša založeni denarni kapital bodisi 422~f.~št. ali pa 500~f.~št., to prav nič ne spremeni dejstva, da nastopa samo kot kapitalska vrednost. \( \KPED' \) ne obstaja več kot kapital, ki je povečal vrednost, ali kot s presežno vrednostjo oplojeni kapital, kot kapitalsko razmerje. Saj mora povečati vrednost šele v procesu. Isto velja za \( \KPEP \KPEpike \KPEP' \); \( \KPEP' \) mora zmeraj delovati in obnavljati krožni tok kot \( \KPEP \), kot kapitalska vrednost, ki mora producirati presežno vrednost. -- V nasprotju s tem se krožni tok blagovnega kapitala ne prične s kapitalsko vrednostjo, ampak s povečano kapitalsko vrednostjo v blagovni obliki, zaradi česar že od začetka ne vključuje le krožnega toka kapitalske vrednosti v blagovni obliki, ampak tudi krožni tok presežne vrednosti. Če poteka enostavna reprodukcija v tej obliki, nastopa \( \KPEB' \) enake velikosti tako na sklepni točki kakor tudi na izhodišču. Če pride del presežne vrednosti v krožni tok kapitala, se sicer pojavi namesto \( \KPEB' \) na koncu \( \KPEB'' \), večji \( \KPEB' \), naslednji krožni tok pa spet začne \( \KPEB' \), kar je samo večji \( \KPEB' \) kakor v prejšnjem krožnem toku, in začne svoj novi krožni tok z večjo akumulirano kapitalsko vrednostjo, torej tudi z razmeroma večjo novo ustvarjeno presežno vrednostjo. V vsakem primeru začne \( \KPEB' \) krožni tok vedno kot blagovni kapital, ki je enak kapitalski vrednosti plus presežni vrednosti.

Kot \( \KPEB \) se \( \KPEB' \) ne pojavi v krožnem toku posameznega industrijskega kapitala kot oblika tega kapitala, ampak kot oblika nekega drugega industrijskega kapitala, če so produkcijska \KPEstran sredstva njegov produkt. Akt \( \KPED \KPEcrta \KPEB \) (se pravi \( \KPED \KPEcrta \KPEPs \)) prvega kapitala je za ta drugi kapital \( \KPEB' \KPEcrta \KPED' \).

V cirkulacijskem aktu \( \KPED \KPEcrta \KPEBrazcepDsPs \) nastopata \( \KPEDs \) in \( \KPEPs \) kot istovetna toliko, kolikor sta oba blago v rokah svojih prodajalcev; tu delavcev, ki prodajajo svojo delovno silo, tam posestnikov produkcijskih sredstev, ki prodajajo ta sredstva. Za kupca, katerega denar deluje pri tem kot denarni kapital, sta blago le, dokler ju ne kupi, dokler nastopata torej nasproti njegovemu kapitalu, ki ima denarno obliko, kot blago. \( \KPEPs \) in \( \KPEDs \) se pri tem razlikujeta samo toliko, kolikor je lahko \( \KPEPs \) v rokah svojega prodajalca enak \( \KPEB' \), torej kapital, medtem ko je \( \KPEDs \) za delavca vedno le blago in postane kapital šele v rokah kupca, kot sestavni del \( \KPEP \).

Zato \( \KPEB' \) ne more nikoli pričeti krožnega toka zgolj kot \( \KPEB \), kot zgolj blagovna oblika kapitalske vrednosti. Kot blagovni kapital pomeni vedno dvoje. Z vidika uporabne vrednosti je produkt delovanja \( \KPEP \), v našem primeru preja, katerega elementi \( \KPEDs \) in \( \KPEPs \), ki so bili prišli iz cirkulacije kot blago, so delovali samo kot tvorci tega produkta. Drugič, z vidika vrednosti pa je kapitalska vrednost \( \KPEP \) plus presežna vrednost \( \KPEpv \), ustvarjena v delovanju \( \KPEP \).

Edinole v krožnem toku \( \KPEB' \) samega se lahko in se mora ločiti \( \KPEB \textrm{ = } \KPEP \textrm{ = } \) kapitalska vrednost od tistega dela \( \KPEB' \), v katerem obstaja presežna vrednost, torej blagovni produkt, v katerem obstaja kapitalska vrednost, od presežnega produkta, v katerem tiči presežna vrednost, bodisi da sta res deljiva, kakor pri preji, ali pa ne, kakor pri stroju. Deljiva postaneta vselej, kakor hitro se \( \KPEB' \) spremeni v \( \KPED' \).

Če se lahko razdeli celotni blagovni produkt v samostojne enakovrstne delne produkte, kakor na primer naših 10.000 funtov preje, in se lahko zato predoči akt \( \KPEB' \KPEcrta \KPED' \) kot vsota zapored opravljenih prodaj, lahko deluje kapitalska vrednost v blagovni obliki kot \( \KPEB \), se lahko loči od \( \KPEB' \), preden se realizira presežna vrednost, torej preden se realizira \( \KPEB' \) v celoti.

Od \KPEstran 10.000 funtov preje za 500~f.~št. je vrednost 8440 funtov = 422~f.~št. = kapitalska vrednost, ločena od presežne vrednosti. Če proda kapitalist najprej 8440 funtov preje za 422~f.~št., pomeni teh 8440 funtov preje \( \KPEB \), vrednost kapitala v blagovni obliki. Presežni produkt 1560 funtov preje = presežna vrednost v višini 78~f.~št., ki jo vrh tega vsebuje \( \KPEB' \), gre v cirkulacijo šele kasneje; kapitalist lahko opravi \( \KPEB \KPEcrta \KPED \KPEcrta \KPEBrazcepDsPs \) pred cirkulacijo presežnega produkta \( \KPEb \KPEcrta \KPEd \KPEcrta \KPEb \).

Ali, če bi kapitalist prodal najprej 7440 funtov preje v vrednosti 372~f.~št., nato pa 1000 funtov preje v vrednosti 50~f.~št., tedaj bi lahko nadomestil s prvim delom \( \KPEB \) produkcijska sredstva (konstantni del kapitala \( \KPEc \)), z drugim delom \( \KPEB \) pa variabilni del kapitala \( \KPEv \), delovno silo, in rezultat bi bil enak.

V primeru takšnih postopnih prodaj in če to dopuščajo pogoji krožnega toka, pa lahko izvede kapitalist, namesto da bi delil \( \KPEB' \) v \( \KPEc \textrm{ + }  \KPEv \textrm{ + } \KPEpv \), to ločitev tudi na alikvotnih delih \( \KPEB' \).

Na primer: 7440 funtov preje = 372~f.~št., ki predstavljajo kot deli \( \KPEB' \) (10.000 funtov preje = 500~f.~št.) konstantni del kapitala, se lahko sami spet dele na 5.535,360 funtov preje v vrednosti 276,768~f.~št., ki nadomestijo samo konstantni del, samo vrednost v 7440 funtih preje porabljenih produkcijskih sredstev; 744 funtov preje v vrednosti 37,200~f.~št., ki nadomeste le variabilni kapital; 1.160,640 funtov preje v vrednosti 58,032~f.~št., ki so kot presežni produkt nosilec presežne vrednosti. Od prodanih 7440 funtov lahko torej nadomesti v njih vsebovano kapitalsko vrednost s prodajo 6.279,360 funtov preje za ceno 313,968~f.~št., vrednost presežnega produkta 1.160,640 funtov = 58,032~f.~št. pa potroši kot dohodek.

Prav tako lahko nadalje deli 1000 funtov preje = 50~f.~št. = variabilno vrednost kapitala in potemtakem proda: 744 funtov preje za 37,200~f.~št., kar je konstantna vrednost kapitala 1000 funtov preje; 100 funtov preje za 5,000~f.~št., kar je variabilni del kapitala istih 1000 funtov preje; torej 844 \KPEstran funtov preje za 42,200~f.~št., kar je nadomestilo v 1000 funtih preje vsebovane vrednosti kapitala; končno 156 funtov preje v vrednosti 7,800~f.~št., ki predstavljajo v njih vsebovani presežni produkt in se kot taki lahko potrošijo.

Slednjič lahko deli, če se prodaja posreči, še preostalih 1560 funtov preje v vrednosti 78~f.~št. tako, da nadomesti prodaja 1.160,640 funtov preje za 58,032~f.~št. vrednost v 1560 funtih preje vsebovanih produkcijskih sredstev, prodaja 156 funtov preje v vrednosti 7,800~f.~št. pa variabilno vrednost kapitala; skupno 1.316,640 funtov preje = 65,832~f.~št., ki nadomeste skupno vrednost kapitala; končno preostane presežni produkt 243,360 funtov = 12,168~f.~št. za porabo kot dohodek.

Tako kot se vsak element \( \KPEc \), \( \KPEv \), \( \KPEpv \), ki ga vsebuje preja, lahko razstavi na enake sestavne dele, se lahko razstavi tudi vsak posamezni funt preje v vrednosti 1 šilinga = 12 penijev.

\[
    \begin{array}{r c c c l}
        \KPEc & \textrm{ = } & \textrm{0,744 funta preje} & \textrm{ = } & \textrm{8,928 penija} \\
        \KPEv & \textrm{ = } & \textrm{0,100 funta preje} & \textrm{ = } & \textrm{1,200 penija} \\
        \KPEpv & \textrm{ = } & \textrm{0,156 funta preje} & \textrm{ = } & \textrm{1,872 penija} \\
        \hline
        \KPEc \textrm{ + } \KPEv \textrm{ + } \KPEpv & \textrm{ = } & \textrm{1 } \phantom{--..} \textrm{funt preje} & \textrm{ = } & \textrm{12 penijev} \\
    \end{array}
\]

Če seštejemo rezultate gornjih treh delnih prodaj, dobimo isti rezultat kot pri istočasni prodaji vseh 10.000 funtov.

\[
    \begin{array}{l}
        \textrm{Pri konstantnem kapitalu imamo:} \\
        \begin{array}{l r c r}
            \textrm{pri prvi prodaji:} & \textrm{5.535,360 funtov preje} & \textrm{ = } & \textrm{276,768~f.~št.} \\
            \textrm{pri drugi prodaji:} & \textrm{744,000 funtov preje} & \textrm{ = }  & \textrm{37,200~f.~št.} \\
            \textrm{pri tretji prodaji:} & \textrm{1.160,640 funtov preje} & \textrm{ = } & \textrm{58,032~f.~št.} \\
            \hline
            \phantom{--} \textrm{skupaj} & \textrm{7.440} \phantom{--} \thinspace \thinspace \thinspace \textrm{funtov preje} & \textrm{=} & \textrm{372} \phantom{--} \thinspace \textrm{~f.~št.} \\
        \end{array} \\
    \end{array}
\]

\[
    \begin{array}{l}
        \textrm{Pri variabilnem kapitalu:} \\
        \begin{array}{l r c r}
            \textrm{pri prvi prodaji:} & \textrm{744,000 funtov preje} & \textrm{ = } & \textrm{37,200~f.~št.} \\
            \textrm{pri drugi prodaji:} & \textrm{100,000 funtov preje} & \textrm{ = } & \textrm{5,000~f.~št.} \\
            \textrm{pri tretji prodaji:} & \textrm{156,000 funtov preje} & \textrm{ = } & \textrm{7,800~f.~št.} \\
            \hline
            \phantom{--} \textrm{skupaj} & \textrm{1.000} \phantom{--} \thinspace \thinspace \thinspace \textrm{funtov preje} & \textrm{ = } & \textrm{50} \phantom{--} \thinspace \textrm{~f.~št.} \\
        \end{array} \\
    \end{array}
\]

\KPEstran

\[
    \begin{array}{l}
        \textrm{Pri presežni vrednosti:} \\
        \begin{array}{l r c r}
            \textrm{pri prvi prodaji:} & \textrm{1.160,640 funtov preje} & \textrm{ = } & \textrm{58,032~f.~št.} \\
            \textrm{pri drugi prodaji:} & \textrm{156,000 funtov preje} & \textrm{ = } & \textrm{7,800~f.~št.} \\
            \textrm{pri tretji prodaji:} & \textrm{243,360 funtov preje} & \textrm{ = } & \textrm{12,168~f.~št.} \\
            \hline
            \phantom{--} \textrm{skupaj} & \textrm{1.560} \phantom{--} \thinspace \thinspace \thinspace \textrm{funtov preje} & \textrm{ = } & \textrm{78} \phantom{--} \thinspace \textrm{~f.~št.} \\
        \end{array} \\
    \end{array}
\]

\[
    \begin{array}{l}
        \textrm{Summa summarum:} \\
        \begin{array}{l r c r}
            \textrm{konstantni kapital:} & \textrm{7.440 funtov preje} & \textrm{ = } & \textrm{372~f.~št.} \\
            \textrm{variabilni kapital:} & \textrm{1.000 funtov preje} & \textrm{ = } & \textrm{50~f.~št.} \\
            \textrm{presežna vrednost:} & \textrm{1.560 funtov preje} & \textrm{ = } & \textrm{78~f.~št.} \\
            \hline
            \phantom{--} \textrm{skupaj} & \textrm{10.000 funtov preje} & \textrm{ = } & \textrm{500~f.~št.} \\
        \end{array} \\
    \end{array}
\]

\( \KPEB' \KPEcrta \KPED' \) ni sam po sebi nič drugega kot prodaja 10.000 funtov preje. Teh 10.000 funtov preje je blago kakor vsaka druga preja. Kupca zanima cena 1 šiling za funt ali 500~f.~št. za 10.000 funtov. Če se spravi kupec pri trgovanju na razčlenjanje vrednostne sestave, stori to le z zahrbtnim namenom dokazati, da bi prodajalec lahko prodal funt preje za manj ko 1 šiling in bi napravil pri tem še vedno dobro kupčijo. Količina, ki jo kupi, pa je odvisna od njegovih potreb; če je na primer lastnik tkalnice, je odvisna od sestave njegovega lastnega kapitala, ki deluje v tkalnici, ne pa od predilčevega, od katerega kupuje. Razmerja, v katerih mora \( \KPEB' \) po eni strani nadomestiti v njem porabljeni kapital (oziroma različne njegove sestavne dele), po drugi strani pa služiti kot presežni produkt bodisi za potrošnjo presežne vrednosti, bodisi za akumulacijo kapitala, obstajajo samo v krožnem toku kapitala, katerega blagovna oblika je 10.000 funtov preje. S prodajo kot tako nimajo nobenega opravka. Razen tega pri tem predpostavljamo, da se proda \( \KPEB' \) po svoji vrednosti, da gre torej le za njegovo spremembo iz blagovne oblike v denarno. Za \( \KPEB' \) kot funkcionalno obliko v krožnem toku tega posameznega kapitala, iz katere se mora nadomestiti produktivni kapital, je seveda odločilno, ali in koliko sta pri prodaji cena in vrednost različni. Toda to nas tu, ko obravnavamo sam\'o oblikovne razlike, ne zanima.

V \KPEstran obliki I, \( \KPED \KPEpike \KPED' \), se pojavlja produkcijski proces sredi med dvema dopolnjujočima se in druga drugi nasprotnima si fazama cirkulacije kapitala; opravljen je, preden pride do sklepne faze \( \KPEB' \KPEcrta \KPED' \). Denar se založi kot kapital, najprej v produkcijske elemente, iz teh se spremeni v blagovni produkt, ta blagovni produkt pa se spremeni spet v denar. To je do kraja sklenjeni kupčijski krog, katerega rezultat je za vse in vsakogar uporabljivi denar. Ponoven začetek je tako možen, ne pa tudi nujen. \( \KPED \KPEpike \KPEP \KPEpike \KPED' \) je prav tako lahko zadnji krožni tok, ki pri izstopu iz podjetja konča funkcijo individualnega kapitala, kakor tudi prvi krožni tok kapitala, ki nanovo stopa v funkcijo. Splošno gibanje je tu \( \KPED \KPEpike \KPED' \), od denarja k več denarja.

V obliki II, \( \KPEP \KPEpike \KPEB' \KPEcrta \KPED' \KPEcrta \KPEB \KPEpike \KPEP ( \KPEP' ) \), sledi celotni proces cirkulacije prvemu \( \KPEP \) in se vrši pred drugim; poteka pa v obratnem redu kakor v obliki I. Prvi \( \KPEP \) je produktivni kapital, njegovo delovanje pa produkcijski proces kot predhodni pogoj naslednjega procesa cirkulacije. Sklepni \( \KPEP \) pa ni produkcijski proces; je le ponovni pojav industrijskega kapitala v njegovi obliki produktivnega kapitala. In sicer je to kot rezultat v zadnji fazi cirkulacije izvedene spremembe vrednosti kapitala v \( \KPEDs \textrm{ + } \KPEPs \), v subjektivne in objektivne faktorje, ki tvorijo s svojo združitvijo eksistenčno obliko produktivnega kapitala. Bodisi kot \( \KPEP \) ali kot \( \KPEP' \) doseže kapital na koncu spet obliko, v kateri mora znova delovati kot produktivni kapital, izvesti produkcijski proces. Splošna oblika gibanja \( \KPEP \KPEpike \KPEP \) je oblika reprodukcije in ne kaže, tako kakor \( \KPED \KPEpike \KPED' \), da je povečanje vrednosti smoter procesa. Zaradi tega še toliko bolj olajšuje klasični ekonomiji, da pušča ob strani posebno kapitalistično obliko produkcijskega procesa in da prikazuje produkcijo kot tako za smoter procesa, kakor da gre samo za to, da se producira čim več in čim ceneje in da se menjava produkt za čim bolj raznovrstne druge produkte, deloma za obnovo produkcije (\( \KPED \KPEcrta \KPEB \)), deloma za konsumpcijo (\( \KPEd \KPEcrta \KPEb \)). Ker nastopata pri tem \( \KPED \) in \( \KPEd \) le kot prehodno cirkulacijsko sredstvo, se lahko spregledajo tako posebnosti denarja kakor tudi denarnega kapitala in je videti celotni proces \KPEstran preprost in naraven, prav kakor je naraven primitivni racionalizem. Tudi pri blagovnem kapitalu se tu pa tam pozabi na profit; tako nastopa, kakor hitro je govor o produkcijskem krožnem toku kot celoti, le kot blago; kakor hitro pa je govor o sestavnih delih vrednosti, nastopa kot blagovni kapital. Seveda se pojavlja potem tudi akumulacija tako kakor produkcija.

V obliki III, \( \KPEB' \KPEcrta \KPED' \KPEcrta \KPEB \KPEpike \KPEP \KPEpike \KPEB' \), začneta obe fazi cirkulacijskega procesa krožni tok, in sicer v istem vrstnem redu kakor v obliki II, \( \KPEP \KPEpike \KPEP \); potem sledi \( \KPEP \), in sicer tako kot v obliki I s svojo funkcijo, produkcijskim procesom; z rezultatom slednjega, z \( \KPEB' \), se krožni tok sklene. Kakor se sklene v obliki II s \( \KPEP \) kot z zgolj ponovnim pojavom produktivnega kapitala, se sklene tu z \( \KPEB' \) kot ponovnim pojavom blagovnega kapitala; kakor mora v obliki II kapital v svoji sklepni podobi \( \KPEP \) znova pričeti proces kot produkcijski proces, se mora s ponovnim nastopom industrijskega kapitala, v obliki blagovnega kapitala, pričeti krožni tok tu znova s cirkulacijsko fazo \( \KPEB' \KPEcrta \KPED' \). Obedve obliki krožnega toka sta nedokončani, ker se ne skleneta z \( \KPED' \), s povečano, nazaj v \emph{denar} spremenjeno vrednostjo kapitala. Obedve se morata torej nadaljevati, zaradi česar vključujeta reprodukcijo. Celotni krožni tok v obliki III je \( \KPEB' \KPEpike \KPEB' \).

To, kar razlikuje tretjo obliko od obeh prvih, je dejstvo, da nastopa samo v tem krožnem toku kot izhodiščna točka povečana vrednost kapitala, ne pa prvotna kapitalska vrednost, ki se mora šele povečati. \( \KPEB' \) kot kapitalsko razmerje je tu izhodišče in kot tak določilno vpliva na celotni krožni tok, ker vključuje tako krožni tok vrednosti kapitala kakor tudi krožni tok presežne vrednosti že v svoji prvi fazi in ker se mora presežna vrednost v enem delu potrošiti kot dohodek, opraviti cirkulacijo \( \KPEb \KPEcrta \KPEd \KPEcrta \KPEb \), v drugem pa delovati kot element akumulacije kapitala, če že ne v vsakem posameznem krožnem toku, pa vsaj v njihovem povprečju.

Oblika \( \KPEB' \KPEpike \KPEB' \) predpostavlja konsumpcijo celotnega blagovnega produkta kot pogoj za normalen potek krožnega toka kapitala. Individualna delavčeva konsumpcija in individualna \KPEstran konsumpcija neakumuliranega dela presežnega produkta obsegata celotno individualno konsumpcijo. Konsumpcija spada torej v celoti -- kot individualna in kot produktivna konsumpcija -- v krožni tok \( \KPEB' \) kot pogoj. Produktivno konsumpcijo (v katero spada po svoji naravi tudi delavčeva individualna konsumpcija, ker je znotraj določenih meja delovna sila stalni produkt delavčeve individualne konsumpcije) opravi vsak posamezni kapital sam. Individualno konsumpcijo -- razen kolikor je nujna za obstoj posameznega kapitalista -- predpostavljamo samo kot družbeni akt, nikakor ne kot akt posameznega kapitalista.

V oblikah I in II se kaže celotno gibanje kot gibanje založene vrednosti kapitala. V obliki III tvori izhodišče kapital v povečani vrednosti v podobi celotnega blagovnega produkta in ima obliko kapitala v gibanju, blagovnega kapitala. Šele ko se spremeni v denar, se razcepi to gibanje v gibanje kapitala in gibanje dohodka. V tej obliki je vključena v krožni tok kapitala tako razdelitev celotnega družbenega produkta kakor tudi posebna razdelitev produkta vsakega posameznega blagovnega kapitala na individualni konsumpcijski sklad na eni strani in na reprodukcijskega na drugi.

\( \KPED \KPEpike \KPED' \) vključuje možnost, da se krožni tok razširi v skladu z velikostjo \( \KPEd \), ki vstopi v obnovljeni krožni tok.

V \( \KPEP \KPEpike \KPEP \) lahko začne \( \KPEP \) novi krožni tok z isto, lahko tudi z manjšo vrednostjo, pa vendar pomeni reprodukcijo v razširjenem obsegu; na primer, če se zaradi povečane produktivnosti dela pocenijo blagovni elementi. Obratno pomeni lahko v nasprotnem primeru, če se na primer produkcijski elementi podražijo, po vrednosti narastli produktivni kapital reprodukcijo v tvarno zoženem obsegu. Isto velja za \( \KPEB' \KPEpike \KPEB' \).

V \( \KPEB' \KPEpike \KPEB' \) je kapital v blagovni obliki predpostavka produkcije; v tem krožnem toku se vrne znova kot predpostavka v drugem \( \KPEB \). Če ta \( \KPEB \) še ni produciran znova ali reproduciran, se krožni tok zaustavi; ta \( \KPEB \) se mora reproducirati, večinoma kot \( \KPEB' \) kakega drugega industrijskega kapitala. V \KPEstran tem krožnem toku je \( \KPEB' \) izhodiščna, prehodna in sklepna točka gibanja; vedno je prisoten. Je trajen pogoj reprodukcijskega procesa.

\( \KPEB' \KPEpike \KPEB' \) se razlikuje od oblik I in II zaradi nekega drugega momenta. Vsem trem krožnim tokom je skupno to, da je oblika, v kateri začne kapital svoj proces krožnega toka, hkrati tudi oblika, v kateri ga sklene, in tako se znajde vnovič v začetni obliki, v kateri znova odpre isti krožni tok. Začetna oblika \( \KPED \), \( \KPEP \), \( \KPEB' \) je vedno oblika, v kateri se založi vrednost kapitala (v III s presežno vrednostjo, ki ji je prirastla), torej z vidika krožnega toka prvotna oblika; sklepna oblika \( \KPED' \), \( \KPEP \), \( \KPEB' \) je vsakokrat spremenjena oblika funkcionalne oblike, ki je v krožnem toku pred njo, ki pa ni prvotna oblika.

Tako je \( \KPED' \) v I spremenjena oblika \( \KPEB' \), sklepni \( \KPEP \) v II spremenjena oblika \( \KPED \) (tako v I kot v II se doseže ta sprememba z navadnim postopkom blagovne cirkulacije, s formalno menjavo mesta med blagom in denarjem); v III je \( \KPEB' \) spremenjena oblika \( \KPEP \), produktivnega kapitala. Toda tu v III se, prvič, sprememba ne nanaša samo na funkcionalno obliko kapitala, ampak tudi na velikost njegove vrednosti; drugič pa sprememba ni rezultat zgolj menjave mesta, katera pripada procesu menjave, ampak resnične spremembe, ki jo doživita uporabna oblika in vrednost blagovnih sestavin produktivnega kapitala v produkcijskem procesu.

Oblika začetne skrajne točke \( \KPED \), \( \KPEP \), \( \KPEB' \) je izhodišče vsakega ustreznega krožnega toka I, II, III; obliko, ki se ponovi v sklepni skrajni točki, poraja in torej določa zaporedje metamorfoz samega krožnega toka. \( \KPEB' \) kot sklepna točka posameznega industrijskega krožnega toka kapitala predpostavlja sam\'o obliko \( \KPEP \), katera ne pripada cirkulaciji istega industrijskega kapitala, ki jo je produciral. \( \KPED' \) kot sklepna točka v I, kot spremenjena oblika \( \KPEB' \) (\( \KPEB' \KPEcrta \KPED' \)), predpostavlja \( \KPED \) v rokah kupca, kakor da obstaja izven krožnega toka \( \KPED \KPEpike \KPED' \) in pride vanj s prodajo \( \KPEB' \) ter postane njegova sklepna oblika. Enako predpostavlja sklepni \( \KPEP \) v II \( \KPEDs \) in \( \KPEPs \) (\( \KPEB \)) kot nekaj, kar obstaja izven njega in kar \KPEstran se mu pripoji z \( \KPED \KPEcrta \KPEB \) kot sklepna oblika. Če odmislimo sklepno skrajno točko, pa vendar ne predpostavlja niti krožni tok individualnega denarnega kapitala obstoja denarnega kapitala nasploh niti krožni tok individualnega produktivnega kapitala obstoja produktivnega kapitala v njegovem krožnem toku. V I je lahko \( \KPED \) prvi denarni kapital, v II je lahko \( \KPEP \) prvi produktivni kapital, ki nastopi na odru zgodovine; v III \[
    \KPEB'
    \left\lbrace
    \begin{array}{c c}
        \KPEB & \KPEcrta \\
        \KPEcrta & \KPED' \\
        \KPEb & \KPEcrta \\
    \end{array}
    \right.
    \left\lbrace
    \begin{array}{c c l}
        \KPED & \KPEcrta & \KPEBrazcepDsPs \KPEpike \KPEP \KPEpike \KPEB' \\
         & & \\
        \KPEd & \KPEcrta & \KPEb \\
    \end{array}
    \right.
\] pa je predpostavljen \( \KPEB \) dvakrat zunaj krožnega toka. Enkrat v krožnem toku \( \KPEB' \KPEcrta \KPED' \KPEcrta \KPEBrazcepDsPs \). Kolikor sestoji iz \( \KPEPs \), je ta \( \KPEB \) blago v rokah prodajalca; če je produkt kapitalističnega produkcijskega procesa, pa je tudi blagovni kapital; toda tudi če ni, nastopa kot blagovni kapital v rokah trgovca. Drugič v drugem \( \KPEb \), v \( \KPEb \KPEcrta \KPEd \KPEcrta \KPEb \), ki mora prav tako obstajati kot blago, da ga je mogoče kupiti. Na vsak način sta \( \KPEDs \) in \( \KPEPs \), če sta blagovni kapital ali pa ne, prav tako blago, kakor je \( \KPEB' \), in se obnašata drug nasproti drugemu kot blago. Isto velja za drugi \( \KPEb \) v \( \KPEb \KPEcrta \KPEd \KPEcrta \KPEb \). Kolikor je torej \( \KPEB' \textrm{ = } \KPEB ( \KPEDs \textrm{ + } \KPEPs ) \), sestoji iz blaga in se mora v cirkulaciji nadomestiti z enakim blagom; kakor se mora tudi v \( \KPEb \KPEcrta \KPEd \KPEcrta \KPEb \) drugi \( \KPEb \) nadomestiti v cirkulaciji z drugim enakim blagom.

Razen tega mora biti, kjer prevladuje kapitalistični produkcijski način, vsako blago v rokah prodajalca blagovni kapital. To je tudi v rokah trgovca ali pa postane v njegovih rokah, če še ni bilo. Ali pa mora biti -- kakor na primer uvoženi predmeti -- takšno blago, ki nadomesti prvotni blagovni kapital, ki mu da torej le drugo obliko obstoja.

Blagovni elementi \( \KPEDs \) in \( \KPEPs \), iz katerih sestoji produktivni kapital \( \KPEP \), nimajo kot eksistenčne oblike \( \KPEP \) iste podobe kakor na različnih blagovnih trgih, od koder so zbrani. Zdaj \KPEstran so združeni in v svoji združitvi lahko delujejo kot produktivni kapital.

To, da nastopa \( \KPEB \) kot predpostavka \( \KPEB \) znotraj samega krožnega toka sam\'o v tej obliki III, izvira iz dejstva, da je izhodišče kapital v blagovni obliki. Krožni tok se prične s pretvorbo \( \KPEB' \) (kolikor deluje kot vrednost kapitala, bodisi že povečana ali nepovečana s presežno vrednostjo) v blago, ki tvori njegove produkcijske elemente. Ta pretvorba pa obsega ves proces cirkulacije \( \KPEB \KPEcrta \KPED \KPEcrta \KPEB ( \textrm{= } \KPEDs \textrm{ + } \KPEPs ) \) in je njegov rezultat. Tu stoji torej \( \KPEB \) na obeh skrajnih točkah; druga skrajna točka, ki dobi svojo obliko \( \KPEB \) s posredovanjem \( \KPED \KPEcrta \KPEB \) od zunaj, z blagovnega trga, pa ni zadnja skrajna točka krožnega toka, ampak le prvih dveh njegovih stadijev, ki vključujeta cirkulacijski proces. Njegov rezultat je \( \KPEP \), ki prične potem delovati, produkcijski proces. Šele kot njegov rezultat, torej ne kot rezultat cirkulacijskega procesa, nastopa \( \KPEB' \) kot zaključek krožnega toka in v isti obliki kot začetna skrajna točka \( \KPEB' \). V nasprotju s tem sta sklepni skrajni točki \( \KPED' \) in \( \KPEP \) v \( \KPED \KPEpike \KPED' \), \( \KPEP \KPEpike \KPEP \) neposredna rezultata cirkulacijskega procesa. Tu se torej predpostavlja, da sta prvič \( \KPED' \), drugič \( \KPEP \) v drugih rokah samo na koncu. Dokler poteka krožni tok med skrajnima točkama, se niti \( \KPED \) (v prvem primeru) niti \( \KPEP \) (v drugem) -- obstoj \( \KPED \) kot tujega denarja, \( \KPEP \) kot tujega produkcijskega procesa -- ne pojavlja kot predpostavka teh krožnih tokov. \( \KPEB' \KPEpike \KPEB' \) pa predpostavlja nasprotno temu \( \KPEB ( \textrm{= } \KPEDs \textrm{ + } \KPEPs ) \) kot tuje blago v tujih rokah, ki ga uvodni cirkulacijski proces pritegne v krožni tok in spremeni v produktivni kapital; kot rezultat njegovega delovanja postane \( \KPEB' \) spet sklepna oblika krožnega toka.

Prav zato, ker predpostavlja krožni tok \( \KPEB' \KPEpike \KPEB' \) v svojem poteku drug industrijski kapital v obliki \( \KPEB ( \textrm{= } \KPEDs \textrm{ + } \KPEPs ) \) (\( \KPEPs \) pa vključuje raznovrstne druge kapitale, v našem primeru npr. stroje, premog, olje itd.), napeljuje sam po sebi, da vidimo v njem ne le \emph{splošno} obliko krožnega toka, to se pravi družbeno obliko, v kateri lahko motrimo vsak posamezni industrijski kapital (razen pri njegovi prvi naložbi), zato ne le obliko gibanja, ki je skupna vsem posameznim industrijskim \KPEstran kapitalom, ampak hkrati obliko gibanja vsote posameznih kapitalov, torej vsega kapitala kapitalističnega razreda, gibanje, v katerem nastopa vsak posamezni industrijski kapital kot delno gibanje, ki se z drugimi prepleta in je od njih odvisno. Če opazujemo na primer celotni letni blagovni produkt kake dežele in razčlenjujemo gibanje, s katerim nadomešča del celotnega blagovnega produkta produktivni kapital v vseh posameznih podjetjih, drugi del pa prehaja v individualno konsumpcijo različnih razredov, opazujemo \( \KPEB' \KPEpike \KPEB' \) kot obliko gibanja tako družbenega kapitala kakor tudi presežne vrednosti oziroma presežnega produkta, ki ga ustvarja. To, da je družbeni kapital enak vsoti posameznih kapitalov (vštevši delniški kapital oziroma državni kapital, kolikor uporabljajo vlade produktivno mezdno delo v rudnikih, na železnicah itd., kolikor delujejo kot industrijski kapitalisti) in da je celotno gibanje družbenega kapitala enako algebrajski vsoti gibanj posameznih kapitalov, nikakor ne izključuje, da kaže to gibanje združenih posameznih kapitalov drugačne pojave kakor isto gibanje, če ga gledamo z vidika enega dela celotnega gibanja družbenega kapitala, torej v njegovi povezanosti z gibanji njegovih drugih delov, in da hkrati rešuje vprašanja, ki se morajo šteti pri proučevanju krožnega toka posameznega individualnega kapitala za rešena, namesto da bi odgovarjalo nanja.

\( \KPEB' \KPEpike \KPEB' \) je edini krožni tok, v katerem je prvotno založena vrednost kapitala samo del skrajne točke, ki pričenja gibanje, in v katerem se najavlja zato gibanje že vnaprej kot totalno gibanje industrijskega kapitala; tako tistega dela produkta, ki nadomesti produktivni kapital, kakor onega, ki pomeni presežni produkt in ki se v povprečju potroši deloma kot dohodek, deloma pa mora služiti za element akumulacije. Kolikor je vključena v ta krožni tok potrošnja presežne vrednosti kot dohodek, toliko je vključena vanj tudi individualna potrošnja. Ta pa je vključena tudi še zato, ker obstaja izhodiščna točka \( \KPEB \), blago, kot tak ali drugačen uporabni predmet; vsak kapitalistično producirani predmet pa je blagovni kapital, najsi ga določa njegova \KPEstran uporabna podoba za produktivno ali za osebno potrošnjo ali za obe. \( \KPED \KPEpike \KPED' \) kaže samo na vrednostno plat, na povečanje vrednosti založene vrednosti kapitala kot smoter vsega procesa; \( \KPEP \KPEpike \KPEP ( \KPEP' ) \) na produkcijski proces kapitala kot reprodukcijski proces z nespremenjeno ali naraščajočo velikostjo produktivnega kapitala (akumulacija); \( \KPEB' \KPEpike \KPEB' \), ki se že v svoji začetni skrajni točki najavlja kot tvorba kapitalistične blagovne produkcije, že prav od začetka obsega produktivno in osebno konsumpcijo; produktivna konsumpcija z večanjem vrednosti, ki ga vključuje, nastopa le kot veja njegovega gibanja. Končno nakazuje dejstvo, da ima \( \KPEB' \) lahko uporabno obliko, ki ne more vnovič vstopiti v noben produkcijski proces, da morajo zavzeti različni sestavni deli \( \KPEB' \), ki jih izražajo deli produkta, različno mesto glede na to, ali je \( \KPEB' \KPEpike \KPEB' \) oblika gibanja celotnega družbenega kapitala ali pa samostojno gibanje posameznega industrijskega kapitala. Vse te njegove posebnosti kažejo, da je ta krožni tok dosti več kakor osamljen krožni tok posameznega kapitala.

V podobi \( \KPEB' \KPEpike \KPEB' \) nastopa gibanje blagovnega kapitala, se pravi kapitalistično produciranega celotnega produkta, tako kot predpostavka samostojnega krožnega toka posameznega kapitala kakor tudi kot njegova posledica. Če hočemo torej razumeti to podobo v njeni posebnosti, ni več dovolj zadovoljiti se z ugotovitvijo, da sta metamorfozi \( \KPEB' \KPEcrta \KPED' \) in \( \KPED \KPEcrta \KPEB \) po eni strani funkcionalno določena odseka metamorfoze kapitala, po drugi strani pa člena splošne blagovne cirkulacije. Nujno postane pojasniti prepletanje metamorfoz enega individualnega kapitala z metamorfozami drugih individualnih kapitalov in s tistim delom skupnega produkta, ki je namenjen osebni potrošnji. Zaradi tega jemljemo pri analizi krožnega toka posameznega industrijskega kapitala za podlago predvsem prvi dve obliki.

Kot oblika posameznega kapitala nastopa krožni tok \( \KPEB' \KPEpike \KPEB' \) na primer v poljedelstvu, kjer računajo od žetve do žetve. Podoba II izhaja torej iz setve, podoba III pa iz žetve ali, kakor pravijo fiziokrati, prvi iz avances [predujmov|, drugi pa iz reprises [izkupičkov]. Gibanje vrednosti \KPEstran kapitala nastopa v III že od početka le kot del gibanja splošne količine produktov, medtem ko je v I in II gibanje \( \KPEB' \) le moment v gibanju posameznega kapitala.

V podobi III je vse blago, ki se nahaja na trgu, stalna predpostavka produkcijskega in reprodukcijskega procesa. Če gledamo na ta lik kot ustaljen, se zaradi tega zdi, kakor da bi vsi elementi produkcijskega procesa izhajali iz blagovne cirkulacije in obstajali samo iz blaga. To enostransko pojmovanje spregleduje elemente produkcijskega procesa, ki niso odvisni od blagovnih elementov.

Ker je v \( \KPEB' \KPEpike \KPEB' \) izhodiščna točka celotni produkt (celotna vrednost), se v njem pokaže (zunanjo trgovino pri tem izvzemamo), da lahko poteka pri sicer nespremenjeni produktivnosti reprodukcija v razširjenem obsegu samo, če tisti del presežnega produkta, ki naj se kapitalizira, že vsebuje tvarne elemente dodatnega produktivnega kapitala; če se torej v primeru, da služi produkcija enega leta za izhodišče produkcije naslednjega, ali v primeru, da je mogoče opraviti to hkrati s procesom enostavne reprodukcije v teku istega leta, presežni produkt takoj producira v obliki, ki mu omogoča, da deluje kot dodatni kapital. Povečana produktivnost lahko pomnoži samo snovnost kapitala, ne poveča pa njene vrednosti; ustvarja pa s tem dodatne tvarine za večanje vrednosti.

\( \KPEB' \KPEpike \KPEB' \) je podlaga Ouesnayevega Tableau \'economique; dejstvo, da je v nasprotju z \( \KPED \KPEpike \KPED' \) (obliko, ki se je izključno drži merkantilistični sistem) izbral to obliko in ne
\( \KPEP \KPEpike \KPEP \), kaže na njegov ostri in pravilni občutek.
 
\end{document}
