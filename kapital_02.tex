\documentclass[egregdoesnotlikesansseriftitles]{scrbook}

\usepackage[utf8]{inputenc}
\usepackage[T1]{fontenc}

\usepackage[russian,slovene]{babel}

\usepackage[stable]{footmisc}

\usepackage{subfiles}

\newcommand{\D}{\textrm{D}}
\newcommand{\crta}{\textrm{---}}
\newcommand{\B}{\textrm{B}}
\newcommand{\pike}{\textrm{\dots}}
\newcommand{\pro}{\textrm{P}}
\newcommand{\Ds}{\textrm{Ds}}
\newcommand{\Ps}{\textrm{Ps}}

\begin{document}

\part{Metamorfoze kapitala in njihov krožni tok}

    \chapter{Krožni tok denarnega kapitala}
    \subfile{poglavje_01}

    \chapter{Krožni tok produktivnega kapitala}
    %\subfile{poglavje_02}

    \chapter{Krožni tok blagovnega kapitala}

    \chapter{Trije liki procesa krožnega toka}

    \chapter{Cirkulacijski čas}

    \chapter{Cirkulacijski stroški}

\part{Obrat kapitala}

    \chapter{Čas obrata in število obratov}

    \chapter{Fiksni in cirkulirajoči kapital}

    \chapter{Celotni obrat založniškega kapitala. Ciklusi obratov}

    \chapter{Teorije o fiksnem in cirkulirajočem kapitalu. Fiziokrati in Adam Smith}

    \chapter{Teorije o fiksnem in cirkulirajočem kapitalu. Ricardo}

    \chapter{Delovno obdobje}

    \chapter{Produkcijski čas}

    \chapter{Cirkulacijski čas}

    \chapter{Vpliv časa obrata na velikost založenega kapitala}

    \chapter{Obrat variabilnega kapitala}

    \chapter{Cirkulacija presežne vrednosti}

\part{Reprodukcija in cirkulacija celotnega družbenega kapitala}

\chapter{Uvod}

    \chapter{Prejšnje razlage predmeta}

    \chapter{Enostavna reprodukcija}

    \chapter{Akumulacija in Razširjena reprodukcija}

\chapter{Imensko kazalo}

\chapter{Stvarno kazalo}

\chapter{Mere in denarne enote}

\chapter{K slovenski izdaji II. zvezka Kapitala}

\end{document}