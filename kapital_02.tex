\documentclass[egregdoesnotlikesansseriftitles, a5paper, DIV=calc]{scrbook}

\usepackage[utf8]{inputenc}
\usepackage[T2A, T1]{fontenc}
\usepackage[russian,slovene]{babel}
\usepackage{microtype}

\usepackage[stable]{footmisc}

\usepackage{subfiles}

\usepackage{tempora}
\recalctypearea

\usepackage{tikz}
\usetikzlibrary{matrix}
\usetikzlibrary{decorations.pathreplacing}
\tikzstyle{overbrace style}=[decorate,decoration={brace,raise=0.25em,amplitude=0.25em}]
\tikzstyle{underbrace style}=[decorate,decoration={brace,raise=0.25em,amplitude=0.25em,mirror}]
\tikzstyle{under underbrace style}=[decorate,decoration={brace,raise=0.75em,amplitude=0.25em,mirror}]

\newcommand{\KPED}{D}
\newcommand{\KPEB}{B}
\newcommand{\KPEP}{P}

\newcommand{\KPEDs}{\mathit{Ds}}
\newcommand{\KPEPs}{\mathit{Ps}}
\newcommand{\KPEPv}{\mathit{Pv}}

\newcommand{\KPEb}{b}
\newcommand{\KPEd}{d}
\newcommand{\KPEp}{p}

%\newcommand{\KPEcrta}{\mathit{\textrm{---}}}
\newcommand{\KPEcrta}{
    \allowbreak
    \begin{tikzpicture}[baseline=-0.5ex, remember picture]
        {
            \draw[-]
                (0,0) --
                (1em,0em);
        }
    \end{tikzpicture}\allowbreak 
}
%\newcommand{\KPEpike}{\ldots}
\newcommand{\KPEpike}{
    \allowbreak
    \begin{tikzpicture}[baseline=-0.2pt, remember picture]
        {
            \draw[fill] (0,0) circle [radius=0.2pt];
            \draw[fill] (0.5em,0) circle [radius=0.2pt];
            \draw[fill] (1em,0) circle [radius=0.2pt];
        }
    \end{tikzpicture}\allowbreak
}
\newcommand{\KPErazcep}{
    \begin{tikzpicture}[baseline=-0.5ex, remember picture]
        {
            \draw[-]
                (0.87em,0.25em) --
                (0,0) --
                (0.87em,-0.5em);
        }
    \end{tikzpicture}
}

\newcommand{\KPEfst}{fst}

\newcommand{\KPEBrazcepDsPs}{\allowbreak\KPEB\KPErazcep^{\KPEDs}_{\KPEPs}\allowbreak}

\begin{document}

%\subfile{guide}

\part{Metamorfoze kapitala in njihov krožni tok}

    \chapter{Krožni tok denarnega kapitala}
    \subfile{./source-tex/poglavje_01}

    %\chapter{Krožni tok produktivnega kapitala}
    %\subfile{./source-tex/poglavje_02}

    %\chapter{Krožni tok blagovnega kapitala}

    %\chapter{Trije liki procesa krožnega toka}

    %\chapter{Cirkulacijski čas}

    %\chapter{Cirkulacijski stroški}

%\part{Obrat kapitala}

    %\chapter{Čas obrata in število obratov}

    %\chapter{Fiksni in cirkulirajoči kapital}

    %\chapter{Celotni obrat založniškega kapitala. Ciklusi obratov}

    %\chapter{Teorije o fiksnem in cirkulirajočem kapitalu. Fiziokrati in Adam Smith}

    %\chapter{Teorije o fiksnem in cirkulirajočem kapitalu. Ricardo}

    %\chapter{Delovno obdobje}

    %\chapter{Produkcijski čas}

    %\chapter{Cirkulacijski čas}

    %\chapter{Vpliv časa obrata na velikost založenega kapitala}

    %\chapter{Obrat variabilnega kapitala}

    %\chapter{Cirkulacija presežne vrednosti}

%\part{Reprodukcija in cirkulacija celotnega družbenega kapitala}

    %\chapter{Uvod}

    %\chapter{Prejšnje razlage predmeta}

    %\chapter{Enostavna reprodukcija}

    %\chapter{Akumulacija in Razširjena reprodukcija}

%\chapter{Imensko kazalo}

%\chapter{Stvarno kazalo}

%\chapter{Mere in denarne enote}

%\chapter{K slovenski izdaji II. zvezka Kapitala}

\end{document}