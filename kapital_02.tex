\documentclass[egregdoesnotlikesansseriftitles, a5paper, DIV=calc]{scrbook}

\usepackage[utf8]{inputenc}
\usepackage[T2A, T1]{fontenc}
\usepackage[russian,slovene]{babel}

\usepackage[stable]{footmisc}

\usepackage{subfiles}

\usepackage{tempora}
\recalctypearea

\newcommand{\D}{D}
\newcommand{\crta}{---}
\newcommand{\B}{B}
\newcommand{\pike}{\dots}
\newcommand{\Pro}{P}
\newcommand{\Ds}{\mathit{Ds}}
\newcommand{\Ps}{\mathit{Ps}}
\newcommand{\Pv}{\mathit{Pv}}
\newcommand{\dB}{b}
\newcommand{\dD}{d}
\newcommand{\dPro}{p}

\begin{document}

\part{Metamorfoze kapitala in njihov krožni tok}

    \chapter{Krožni tok denarnega kapitala}
    \subfile{./source-tex/poglavje_01}

    \chapter{Krožni tok produktivnega kapitala}
    \subfile{./source-tex/poglavje_02}

    \chapter{Krožni tok blagovnega kapitala}

    \chapter{Trije liki procesa krožnega toka}

    \chapter{Cirkulacijski čas}

    \chapter{Cirkulacijski stroški}

\part{Obrat kapitala}

    \chapter{Čas obrata in število obratov}

    \chapter{Fiksni in cirkulirajoči kapital}

    \chapter{Celotni obrat založniškega kapitala. Ciklusi obratov}

    \chapter{Teorije o fiksnem in cirkulirajočem kapitalu. Fiziokrati in Adam Smith}

    \chapter{Teorije o fiksnem in cirkulirajočem kapitalu. Ricardo}

    \chapter{Delovno obdobje}

    \chapter{Produkcijski čas}

    \chapter{Cirkulacijski čas}

    \chapter{Vpliv časa obrata na velikost založenega kapitala}

    \chapter{Obrat variabilnega kapitala}

    \chapter{Cirkulacija presežne vrednosti}

\part{Reprodukcija in cirkulacija celotnega družbenega kapitala}

\chapter{Uvod}

    \chapter{Prejšnje razlage predmeta}

    \chapter{Enostavna reprodukcija}

    \chapter{Akumulacija in Razširjena reprodukcija}

\chapter{Imensko kazalo}

\chapter{Stvarno kazalo}

\chapter{Mere in denarne enote}

\chapter{K slovenski izdaji II. zvezka Kapitala}

\end{document}